\chapter[短章名]{这是一个充分长的章名, 强势占用三行没商量.}
	如果章名过长, 可以在目录和天眉以另外设置的短章名显示, 方式和 \LaTeX 的标准文档类 \textsf{book} 相同.

	\section{长度正常的节名}
	复数 $\tau$ 的虚部记为 $\Im(\tau)$. 自然对数记为 $\log$.
	
	\begin{convention}
		本节记 Poincaré 上半平面为
		\[ \mathcal{H} := \left\{ \tau \in \mathbb{C}: \Im(\tau) > 0 \right\}. \]
		按例记 $q := e^{2\pi i \tau}$.	\emph{Dedekind $\eta$ 函数}定义为无穷乘积
		\[ \eta(\tau) := e^{2\pi i \tau/24} \prod_{n=1}^\infty (1 - q^n), \quad \tau \in \mathcal{H}. \]
	\end{convention}
	由分析学常识易见此无穷乘积绝对收敛. 进一步, $\eta$ 在 $\mathcal{H}$ 上全纯无零点; 此外 $\eta$ 的对数导数为
	\[ \frac{\mathrm{d}}{\mathrm{d}\tau} \log \eta (\tau) := \frac{\eta'(\tau)}{\eta(\tau)} = \frac{\pi i}{12} - 2\pi i \sum_{n=1}^\infty \frac{n q^n}{1 - q^n}. \]
	
	\begin{example}
		在右半复平面上定义 $\sqrt{z} := \exp(\log|z| + i\arg(z))$, 其中幅角取 $\arg(z) \in \left[-\frac{\pi}{2}, \frac{\pi}{2}\right]$. 则
		\[ \eta\left( \frac{-1}{\tau} \right) = \sqrt{-i\tau} \cdot \eta(\tau), \quad \tau \in \mathcal{H}. \]
	\end{example}
	\begin{proof}
		应用 Eisenstein 级数 $E_2$ 的性质, 将对数导数 $\frac{\mathrm{d}}{\mathrm{d} \tau} \log\eta(\tau)$ 整理为
		\begin{multline*}
			\frac{\pi i}{12} - 2\pi i \sum_{d \geq 1} \frac{dq^d}{1- q^d} = \frac{\pi i}{12} - 2\pi i \sum_{d \geq 1} \sum_{k \geq 1} d q^{dk} \\
			\stackrel{n := dk}{=} \frac{\pi i}{12} - 2\pi i \sum_{n \geq 1} \sigma_1(n) q^n = \dfrac{\pi i}{12} \cdot E_2(\tau),
		\end{multline*}
		若改为对 $\tau \mapsto \eta(\frac{-1}{\tau})$ 求对数导数, 再应用 $E_2$ 的函数方程, 产物则是
		\[ \tau^{-2} \cdot \frac{\pi i}{12} \cdot E_2\left(\frac{-1}{\tau}\right) = \frac{\pi i}{12} \left(E_2(\tau) + \frac{12}{2\pi i \tau} \right). \]
		对 $\sqrt{-i\tau}$ 求对数导数给出 $\frac{1}{2} \frac{\mathrm{d}}{\mathrm{d}\tau} \log(-i\tau) = \dfrac{1}{2\tau} = \dfrac{\pi i}{12} \cdot \dfrac{12}{2 \pi i \tau}$. 与上式对比即见
		\begin{align*}
			\frac{\mathrm{d}}{\mathrm{d} \tau} \log \eta\left( \frac{-1}{\tau} \right) & = \frac{\mathrm{d}}{\mathrm{d} \tau} \log \sqrt{-i\tau} + \frac{\mathrm{d}}{\mathrm{d} \tau} \log \eta(\tau)  \\
			& = \frac{\mathrm{d}}{\mathrm{d} \tau} \log \left( \sqrt{-i\tau} \cdot \eta(\tau) \right).
		\end{align*}
		故存在 $c \in \mathbb{C}^\times$ 使得 $\eta\left(\frac{-1}{\tau}\right) = c \sqrt{-i\tau} \cdot \eta(\tau)$; 因为 $\eta(i) \neq 0$, 代入 $\tau = i$ 可知 $c = 1$.
	\end{proof}

	著名的 Euler 五边形数定理写作
	\begin{equation}\label{eqn:pentagonal-number}
		\sum_{n \in \mathbb{Z}} (-1)^n q^{(3n^2 + n)/2} = \prod_{n \geq 1} (1 - q^n);
	\end{equation}
	留意到 $3n^2 + n \equiv 0 \pmod 2$ 恒成立. 将 $\frac{3n^2 + n}{2} = \frac{(6n+1)^2 - 1}{24}$ 代入 \eqref{eqn:pentagonal-number}, 即可导出 $\eta$ 的 Fourier 展开
	\begin{equation*}
		\eta(\tau) = \sum_{n \in \mathbb{Z}} (-1)^n q^{ \frac{1}{24} \cdot (6n + 1)^2}, \quad q^{1/24} := e^{2\pi i \tau /24}.
	\end{equation*}

	\section[短节名]{
		四十男儿学干谒,朝游江淮暮吴越。漫将衣食累朱门,讵有文章动金阙。
		倦游屡岁赋归欤,故人相值还唏嘘。劝我莫作千里客,留我共读三冬书。
		忆别吴阊一年久,为我糟床压春酒。入座争迎作赋才,当筵更觅弹筝手。
		酒酣慷慨唤奈何,风光一往如流波。女坟湖北莺犹少,短簿祠南雨正多。
		君家奇书一千轴,锦袱牙签光历碌。愿随潘左伴青缃,羞与金张斗华毂。
		嗟余短鬓日苍浪,太息忧来未可忘。鼓挝马槊差亦得,若问读书非我长。}
	原诗作者: [清] 陈维崧.

	不鼓励使用过长的节名. 同样地, 可以在目录和天眉以另外设置的短节名显示, 方式和 \LaTeX 的标准文档类 \textsf{book} 相同.

	\begin{Exercises}
		\item 造访兰州. \begin{hint} 低碳出行, 请乘坐火车. \end{hint}
		\item 造访祁连山.
	\end{Exercises}