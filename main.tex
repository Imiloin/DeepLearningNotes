%!TEX TS-program = xelatex
%!TEX encoding = UTF-8

% 使用自定义的文档类 AJbook.cls. 自动载入 xeCJK. 需要额外档案如下:
% font-setup-open.tex, coverpage.tex, titles-setup.tex, mycommand.sty, myarrows.sty
% 文档类选项 (key/val 格式):
% draftmark = true (未定稿, 底部显示日期) 或 false (成品), 默认 false,
% colors = true (链接带颜色无框) 或 false (黑色无框), 默认 true,
% traditional = true (繁体中文) 或 false (简体中文), 默认 false,
% coverpage = 封面档档名, 默认为空 (此时不制作封面), 一般是 .tex 档, 若为 *.pdf 的形式则直接引入 PDF 页面.
% fontsetup = 字体设置档档名,
% titlesetup = 章节格式设置档名.

% 注意: 如果文中未使用 \cite 和 \index 命令, 则可能报错.

% 需动用 imakeidx + xindy (两份索引), biblatex + biber (书目). 
% 索引用土法进行中文排序: 如 \index{zhongwen@中文} 等.
\documentclass[
	draftmark = true,   % 草稿模式下, 每页底部将打印相关版本信息.
%	traditional = true,
%	colors = false,
	coverpage = cover.pdf,
    geometry = a4,    % 版面设置, 目前仅容许 a4, b5, 默认 b5, 其它字串则不作自动设置.
	fontsetup = font-setup-open.tex,
	titlesetup = titles-setup.tex
]{AJbook}

\usepackage{bm}  % 数学粗体
\usepackage{mathrsfs}
\usepackage{stmaryrd} \SetSymbolFont{stmry}{bold}{U}{stmry}{m}{n}	% 避免警告 (stmryd 不含粗体故)
\usepackage{array}
\usepackage{makecell}	% 便于制表
\usepackage{tikz-cd}  % 使用 TikZ 绘图
\usetikzlibrary{positioning, patterns, calc, matrix, shapes.arrows, shapes.symbols}
\usepackage{braids}
\usepackage{tqft}
\usepackage{ytableau}
\usepackage{multirow}
\usepackage{threeparttable}	% 表格注释
\usepackage[inkscapearea=page]{svg}	% 插入 svg 图形
\graphicspath{{imgs/}}	% 设定图片目录
\usepackage{pythonhighlight}

% PGF plots 用于封面绘制
\usepackage{pgfplots}
\pgfplotsset{compat=newest}

% 可以修改章节编号的深度
% \setcounter{secnumdepth}{3}

% 必要时仅引入部分档案
% \includeonly{}

% 生成索引: 选用 xindy + imakeidx
\usepackage[xindy, splitindex]{imakeidx}
\makeindex[
	columns=2,
	program=truexindy,
	intoc=true,
	options=-M texindy -I xelatex -C utf8,
	title={Keyword Index}]	% 名词索引

\usepackage[unicode, bookmarksnumbered]{hyperref}	% 启动超链接和 PDF 文档信息所需
% 设置 PDF 文件信息
\hypersetup{
	pdfauthor = {Imiloin},
	pdftitle = {AJbook 文档类模板},
	pdfkeywords = {Template},
	CJKbookmarks = true}

% 增加表格高度
\renewcommand*\arraystretch{1.5}

% 自订公式的编号 (非必要)
\numberwithin{equation}{section}
\renewcommand{\theequation}{\thesection--\arabic{equation}}

% 自订 figure 的编号 (非必要)
%\numberwithin{figure}{section}
%\renewcommand{\thefigure}{\thechapter--\arabic{figure}}

% 自订 table 的编号 (非必要)
%\numberwithin{table}{section}
%\renewcommand{\thetable}{\thechapter--\arabic{table}}

% 用 bibLaTeX 生成参考文献
% 载入书目库: 文档类业已引入 biblatex + biber
\usepackage[backend=biber]{biblatex}
\addbibresource{references.bib}

\begin{document}
	\frontmatter	% 制作封面和目录.
	% 注意: 为了简化, 本模板不含封面. 请通过文档类的参数进行设置.
	
	\mainmatter		% 正文开始:逐章引入 TeX 代码

	\chapter*{Introduction}
	本文档为 \nolinkurl{Deeplearning.ai} 开设的\href{https://www.coursera.org/specializations/deep-learning}{深度学习专项课程}的个人笔记。\LaTeX 模板来自李文威老师的\href{https://github.com/wenweili/AlJabr-1}{《代数学方法》开源项目},做了一些小的修改。封面设计参考了 \href{https://dribbble.com/shots/17836795-Minimal-posters}{Nicolas Solerieu 在 Dribble 上的设计}。

	本项目遵循 \href{https://creativecommons.org/licenses/by-nc-sa/4.0/}{CC BY-NC-SA 4.0} 协议。
	\vspace{1em}
	\begin{flushright}\begin{minipage}{0.2 \textwidth}
		\begin{tabular}{c}
			{By Imiloin} \\
			\href{https://github.com/Imiloin}{Github profile}\\
		\end{tabular}
	\end{minipage}\end{flushright}


	% % % % % % % % % %
	\part{Course Notes}

	\chapter{Neural Networks and Deep Learning}

\section{Introduction to Deep Learning}
机器学习本质上就是训练模型来完成\textbf{从输入到输出($\bm{X\to Y}$)的映射}。例如, 给定一张猫的照片, 机器学习模型可以输出``这是一只猫''的结论。

\vspace{0.5\baselineskip} % 半行距

深度学习有3大要点:
\begin{itemize}	
	\item Data
	\item Computation
	\item Algorithms
\end{itemize}

Data可以分为Structured Data和Unstructured Data。Structured Data是指有固定格式的数据, 例如表格、数据库等。Unstructured Data是指没有固定格式的数据, 例如图片、音频、视频等。

Computation是指计算机的计算能力, 例如CPU、GPU、TPU等。

Algorithms是指算法, 例如Logistic Regression、SVM、Neural Networks等。

\section{Neural Networks Basics}

\subsection{Binary Classification}

表 \ref{tab:notations} 是本章用到的符号说明。

\begin{table}[h]
	\centering
	\begin{threeparttable}
	%%%%%
	\caption{表格标题}
	%
	\begin{tabular}{clcc}
		\hline
									& \textbf{Notation}                         & \textbf{Description} & \textbf{Meaning}                                                   \\ \hline
		\multirow{5}{*}{Sizes}      & $m$                                       & value                & 样本容量                                                               \\
									& $n_x$                                     & value                & 单个样本的特征数(输入)                                                \\
									& $n_y$                                     & value                & 单个样本的标签数(输出)                                                \\
									& $n_h^{[l]}$                               &                      &                                                                    \\
									& $L$                                       &                      &                                                                    \\ \hline
		\multirow{9}{*}{Objects}    & $x^{(i)} \in \mathbb{R}^{n_x}$            & vector               & 第$i$个样本数据                                                          \\
									& $X \in {\mathbb{R}^{n_x \times m}}$       & matrix               & 输入矩阵                                                               \\
									& $a^{(i)} / y^{(i)} \in \mathbb{R}^{n_y}$  & vector               & 第$i$个样本的标签                                                         \\
									& $Y \in {\mathbb{R}^{n_y \times m}}$       & matrix               & 标签矩阵($n_y=1$时退化为行向量$y$)                                    \\									
									& $\hat{y}^{(i)} \in \mathbb{R}^{n_y}$      & vector               & 第$i$个样本的标签预测值向量                                                \\
									& $\hat{Y} \in {\mathbb{R}^{n_y \times m}}$ & matrix               & 标签预测值矩阵($n_y=1$时退化为行向量$\hat{y}$)   						\\
									& $W \in \mathbb{R}^{n_x \times n_y}$       & matrix               & 权重矩阵($n_y=1$时退化为列向量$w$)                                    \\
									& $b \in \mathbb{R}^{n_y}$                  & vector               & 标签偏置值向量                                                        \\
									& $W^{[l]}$                                 &                      &                                                                    \\
									& $b^{[l]}$                                 &                      &                                                                    \\ \hline
		\multirow{2}{*}{Other}      & $J(x,W,b,y)$ or $J(\hat{y},y)$            & function             & 代价函数                                                               \\
									& $\mathrm{d}\mathrm{var}$                  & differential         & 代价函数对变量$\mathrm{var}$的微分,即${\mathrm{d}J}/{\mathrm{d}\mathrm{var}}$ \\ \hline
	\end{tabular}
	%
	\label{tab:notations} %%%
	\begin{tablenotes}
		\item[*] 通常情况下$(i)$表示第$i$个样本,而$[l]$表示第$l$层。
	\end{tablenotes}
	%%%%%
	\end{threeparttable}
\end{table}

\subsection{Logistic Regression}

对于\textbf{二元分类模型,$\bm{n_y=1}$},相关公式如下:
\begin{equation}
	\hat{y}^{(i)} = \sigma(w^\mathrm{T} x^{(i)} + b) \label{eq:logistic}
\end{equation}
其中sigmoid函数$\sigma(z)$定义为:
\begin{equation}
	\sigma(z) = \frac{1}{1 + \mathrm{e}^{-z}} \label{eq:sigmoid}
\end{equation}
该函数可以把实数域$\mathbb{R}$映射到区间$(0, 1)$,如图 \ref{fig:sigmoid} 所示。而$z = w^\mathrm{T} x + b$为线性函数,$w$为权重(weight),$b$为偏置(bias)。
\begin{figure}[h]
	\centering
	\includesvg[width=8cm]{sigmoid}
	\caption{Sigmoid 函数,关于点(0, 0.5)对称}
	\label{fig:sigmoid}
\end{figure}

\vspace{0.5\baselineskip}
对某一个样本而言,Loss function(误差函数)定义为:
\begin{equation}
	L(\hat{y}^{(i)}, y^{(i)}) = -\left[y^{(i)} \log \hat{y}^{(i)} + (1 - y^{(i)}) \log (1 - \hat{y}^{(i)})\right] \label{eq:loss}
\end{equation}
该函数与方差类似,当样本的误差越大,Loss function的值越大,如图 \ref{fig:loss} 所示。
\begin{figure}[h]
	\centering
	\includesvg[width=8cm]{LossFunction}
	\caption{Loss Function 误差函数}
	\label{fig:loss}
\end{figure}

对整个模型而言,Cost function(代价函数)定义为:
\begin{equation}
	J(w, b) = \frac{1}{m} \sum_{i=1}^{m} L(\hat{y}^{(i)}, y^{(i)}) = -\frac{1}{m} \sum_{i=1}^{m} \left[y^{(i)} \log \hat{y}^{(i)} + (1 - y^{(i)}) \log (1 - \hat{y}^{(i)})\right] \label{eq:cost}
\end{equation}
相当于误差函数的平均值。

\subsection{Gradient Descent}

训练过程使用梯度下降(Gradient Descent)算法,即:
\begin{equation}
	\mathrm{var} = \mathrm{var} - \alpha \frac{\mathrm{d}J}{\mathrm{d}\mathrm{var}} \label{eq:gradient}
\end{equation}

其中$\mathrm{var}$为一个需要调整的参数,$\alpha$为学习率(Learning Rate)。对于Logistic Regression,梯度下降算法的表达式为:
\begin{equation}
	\begin{aligned}
	w &:= w - \alpha \frac{\mathrm{d}J}{\mathrm{d}w} \\
	b &:= b - \alpha \frac{\mathrm{d}J}{\mathrm{d}b}
	\end{aligned} 
	\label{eq:gradient_logistic}
\end{equation}
其中$:=$表示赋值,通过不断更新$w$, $b$的值使$J$尽可能小。下面介绍具体实现

\vspace{0.5\baselineskip}
对于一个样本,有
\begin{equation}
	\begin{aligned}
	z &= w^\mathrm{T} x^{(i)} + b \\
	\hat{y}^{(i)} &= \sigma(z) \\
	L(\hat{y}^{(i)}, y^{(i)}) &= -\left[y^{(i)} \log \hat{y}^{(i)} + (1 - y^{(i)}) \log (1 - \hat{y}^{(i)})\right]
	\end{aligned} 
	\label{eq:gradient_logistic_sample}
\end{equation}
进行求导($x$, $y$均为常量),有
\begin{equation}
	\begin{aligned}
	&\frac{\mathrm{d}L}{\mathrm{d}\hat{y}^{(i)}} = -\frac{y^{(i)}}{\hat{y}^{(i)}} + \frac{1 - y^{(i)}}{1 - \hat{y}^{(i)}} \\
	&\frac{\mathrm{d}\hat{y}^{(i)}}{\mathrm{d}z} = \frac{\mathrm{e}^{-z}}{(1 + \mathrm{e}^{-z})^2} = \frac{1}{1 + \mathrm{e}^{-z}}(1 - \frac{1}{1 + \mathrm{e}^{-z}}) = \hat{y}^{(i)}(1 - \hat{y}^{(i)})\\
	\end{aligned}
\end{equation}
用链式法则,有
\begin{equation}
	\frac{\mathrm{d}L}{\mathrm{d}z} = \frac{\mathrm{d}L}{\mathrm{d}\hat{y}^{(i)}} \frac{\mathrm{d}\hat{y}^{(i)}}{\mathrm{d}z} = (-\frac{y^{(i)}}{\hat{y}^{(i)}} + \frac{1 - y^{(i)}}{1 - \hat{y}^{(i)}})\hat{y}^{(i)}(1 - \hat{y}^{(i)}) = \hat{y}^{(i)} - y^{(i)}
\end{equation}
进而有
\begin{equation}
	\begin{aligned}
	\frac{\mathrm{d}L}{\mathrm{d}w_j} &= \frac{\mathrm{d}L}{\mathrm{d}z} \frac{\mathrm{d}z}{\mathrm{d}w_j} = x_j^{(i)} (\hat{y}^{(i)} - y^{(i)}) \quad (1 \leqslant j \leqslant n_x) \\
	\frac{\mathrm{d}L}{\mathrm{d}b} &= \frac{\mathrm{d}L}{\mathrm{d}z} \frac{\mathrm{d}z}{\mathrm{d}b} = \hat{y}^{(i)} - y^{(i)}
	\end{aligned}
\end{equation}

\vspace{0.5\baselineskip}
对于整个训练集,在一个循环中我们对每一个样本都进行一次梯度下降,并对所有样本的梯度进行平均,有
\begin{equation}
	\begin{aligned}
		\frac{\mathrm{d}J}{\mathrm{d}w_j} &= \frac{1}{m} \sum_{i=1}^{m} \frac{\mathrm{d}L}{\mathrm{d}w_j} = \frac{1}{m} \sum_{i=1}^{m} x_j^{(i)} (\hat{y}^{(i)} - y^{(i)}) \quad (1 \leqslant j \leqslant n_x)\\
		\frac{\mathrm{d}J}{\mathrm{d}b} &= \frac{1}{m} \sum_{i=1}^{m} \frac{\mathrm{d}L}{\mathrm{d}b} = \frac{1}{m} \sum_{i=1}^{m} (\hat{y}^{(i)} - y^{(i)})
	\end{aligned}
\end{equation}
得到梯度的平均值后,再进行参数更新,即
\begin{equation}
	\begin{aligned}
	w_j &:= w_j - \alpha \frac{\mathrm{d}J}{\mathrm{d}w} \quad (1 \leqslant j \leqslant n_x) \\
	b &:= b - \alpha \frac{\mathrm{d}J}{\mathrm{d}b}
	\end{aligned} 
\end{equation}
这样就完成了一轮梯度下降的循环。重复多次梯度下降,直到$J$收敛或达到最大迭代次数,就得到了最终的$w$, $b$。

\subsection{Vectorization}

一个重要的原则是,在训练过程中要避免使用循环,而是使用向量化的方法。向量化可以调用并行计算,相对于循环的串行计算效率更高。\\
要完成下面的矩阵操作,
\begin{equation}
	Z =
	\begin{bmatrix}
		z^{(1)} & z^{(2)} & \cdots & z^{(m)}
	\end{bmatrix}
	= w^\mathrm{T} X + 
	\begin{bmatrix}
		b & b & \cdots & b
	\end{bmatrix}
\end{equation}
可以使用下面的代码
\begin{python}
import numpy as np
# define X, Y, w, b
Z = np.dot(w.T, X) + b
\end{python}
同理,要实现
\begin{equation}
	\begin{aligned}
		\frac{\mathrm{d}J}{\mathrm{d}w_j} &= \frac{1}{m} \sum_{i=1}^{m} \frac{\mathrm{d}L}{\mathrm{d}w_j} = \frac{1}{m} \sum_{i=1}^{m} x_j^{(i)} (\hat{y}^{(i)} - y^{(i)}) \quad (1 \leqslant j \leqslant n_x)\\
		\frac{\mathrm{d}J}{\mathrm{d}b} &= \frac{1}{m} \sum_{i=1}^{m} \frac{\mathrm{d}L}{\mathrm{d}b} = \frac{1}{m} \sum_{i=1}^{m} (\hat{y}^{(i)} - y^{(i)})
	\end{aligned}
\end{equation}
可以继续使用下面的代码
\begin{python}
def sigmoid(x):
    s = 1 / (1 + np.exp(-x))
    return s


A = sigmoid(Z)
dZ = A - Y
dw = np.dot(X, dZ.T) / m
# sum along the column
db = np.sum(dZ, axis=1) / m
# sum will get a row vector regardless of the axis parameter, so we need to reshape it
db = db.reshape(ny, 1)
\end{python}
\verb|np.sum()| 函数的 \verb|axis| 参数表示沿着哪个维度进行求和,\verb|axis=0| 表示沿着列求和,\verb|axis=1| 表示沿着行求和。

\vspace{0.5\baselineskip}
在 \verb|numpy| 中,矩阵的维度扩充与 \verb|matlab| 类似,可以自动扩充,也可以使用 \verb|reshape| 函数进行扩充。

矩阵(或向量)在保存时,使用了两重方括号,例如 \verb|[[1, 2], [3, 4]] / [[3, 6, 9]]|,而数组仅有一重方括号,例如 \verb|[1, 2, 3, 4]|。
要把数组转换为向量,可以使用 \verb|reshape()| 函数,例如
\begin{python}
A = np.array([1, 2, 3, 4])
print(A)
print(A.shape)
print(A.T)
A = A.reshape(1, 4)
print(A)
print(A.shape)
print(A.T)
\end{python}
输出结果为
\begin{python}
# array, rank=1
[1 2 3 4]
(4,) 
[1 2 3 4]
# vector
[[1 2 3 4]]
(1, 4) 
[[1]
 [2]
 [3]
 [4]]
\end{python}
\verb|reshape()| 是一个$O(1)$的操作,不会影响程序的运行效率。在本课程中不要使用 \verb|rank 1 array|,因为它的维度不是$(n, 1)$或$(1, n)$,在进行矩阵运算时会出现错误。

\vspace{0.5\baselineskip}
为了确保矩阵的维度正确,可以使用 \verb|assert()| 函数,例如
\begin{python}
assert(A.shape == (1, 4))
\end{python}
如果 \verb|A.shape| 不等于 \verb|(1, 4)|,则会报错。

\subsection{Explanation of logistic regression cost function}

对于二元分类模型,我们期望$\hat{y}$尽可能接近$y$,即$\hat{y}^{(i)}$接近$1$时,$y^{(i)}$也接近$1$,$\hat{y}^{(i)}$接近$0$时,$y^{(i)}$也接近$0$。

当$y^{(i)}=1$时,得到正确分类的概率为$\hat{y}^{(i)}$;而当$y^{(i)}=0$时,得到正确分类的概率为$1-\hat{y}^{(i)}$。因此,对于第$i$个样本,可以使用下面的公式来表示这个概率:
\begin{equation}
	P(y^{(i)} | x^{(i)}; w, b) = (\hat{y}^{(i)})^{y^{(i)}} (1-\hat{y}^{(i)})^{1-y^{(i)}}
\end{equation}
假如各个样本是独立的,那么所有样本得到正确分类的概率为
\begin{equation}
	\begin{aligned}
		P(y | x; w, b) &= \prod_{i=1}^{m} P(y^{(i)} | x^{(i)}; w, b)\\
		&= \prod_{i=1}^{m} (\hat{y}^{(i)})^{y^{(i)}} (1-\hat{y}^{(i)})^{1-y^{(i)}}
	\end{aligned}
\end{equation}
为了避免使用指数运算,可以在等式两边取对数,得到
\begin{equation}
	\begin{aligned}
		\log P(y | x; w, b) &= \sum_{i=1}^{m} \log (\hat{y}^{(i)})^{y^{(i)}} (1-\hat{y}^{(i)})^{1-y^{(i)}}\\
		&= \sum_{i=1}^{m} \left[ y^{(i)} \log \hat{y}^{(i)} + (1-y^{(i)}) \log (1-\hat{y}^{(i)}) \right]
	\end{aligned}
\end{equation}
习惯上等式右边要取平均值,因此可以得到
\begin{equation}
	\frac{1}{m} \log P(y | x; w, b) = \frac{1}{m} \sum_{i=1}^{m} \left[ y^{(i)} \log \hat{y}^{(i)} + (1-y^{(i)}) \log (1-\hat{y}^{(i)}) \right]
\end{equation}
我们期望这个值越大越好,为了使用梯度下降法,可以对上式取负号,得到
\begin{equation}
	J(w, b) \triangleq - \frac{1}{m} \sum_{i=1}^{m} \left[ y^{(i)} \log \hat{y}^{(i)} + (1-y^{(i)}) \log (1-\hat{y}^{(i)}) \right]
\end{equation}
这就是代价函数的定义。我们期望代价函数的值越小越好,因此可以使用梯度下降法来求解$w$和$b$。


	% % % % % % % % % %
	\part{Projects}
	
	\chapter[Improving Deep Neural Networks]{Improving Deep Neural Networks\setcounter{footnote}{0}\footnote{Hyperparameter Tuning, Regularization and Optimization}}

%%%
\section{Setting up and Regularizing neural network}

在本章中会引入一些新的符号,如表~\ref{tab:notations-chap2} 所示。

\begin{table}[htb!]
    \centering
    \begin{threeparttable}
    %%
    \caption{Appended Notations in Chapter 2}
    %
    \begin{tabular}{clcc}
        \toprule
                                    & \textbf{Notation}                                                                     & \textbf{Description} & \textbf{Meaning}                                                   \\ 
        \midrule
        \multirow{1}{*}{Sizes}      & $\verb|batch_size| \in \mathbb{Z}^+$								                    & value                & 每个 \verb|mini_batch| 使用的样本数          \\
        \midrule
        \multirow{5}{*}{Objects}    & $X^{\{t\}} \in \mathbb{R}^{n_x \times \mathtt{batch\_size}}$                          & matrix               & 第$t$个 \verb|mini_batch| 的输入矩阵                                                               \\
                                    & $Y^{\{t\}} \in {\mathbb{R}^{n_y \times \mathtt{batch\_size}}}$                        & matrix               & 第$t$个 \verb|mini_batch| 的样本标签矩阵                                                           \\									
                                    & $\hat{Y}^{\{t\}} / A^{[L]\{t\}} \in {\mathbb{R}^{n_y \times \mathtt{batch\_size}}}$   & matrix               & 第$t$个 \verb|mini_batch| 的标签预测值矩阵   						\\
                                    & $v_{\d \mathrm{var}}$                                                                 & matrix               & 梯度 $\d \mathrm{var}$ 第一矩估计的移动平均值                                 \\
                                    & $s_{\d \mathrm{var}}$                                                                 & matrix               & 梯度 $\d \mathrm{var}$ 第二矩估计的移动平均值                                 \\
        \midrule
        \multirow{6}{*}{Other}      & $\lambda \in \mathbb{R}^+_0$                                                          & value                & 正则化参数                                                             \\  
                                    & \verb|keep_prob| $\in [0,1]$                                                          & value                & Dropout算法的保留概率                                                      \\ 
                                    & $J^{\{t\}}$                                                                           & function             & 第$t$个 \verb|mini_batch| 的代价函数                                                                        \\  
                                    & $\beta_1 \in [0,1]$                                                                   & value                & 第一矩估计的衰减率 \\ 
                                    & $\beta_2 \in [0,1]$                                                                   & value                & 第二矩估计的衰减率                                                    \\ 
                                    & $\varepsilon \in \mathbb{R}^+$                                                        & value                & 第二矩估计中的数值稳定性参数                                                    \\
        \bottomrule
    \end{tabular}
    %
    \label{tab:notations-chap2} %
    \begin{tablenotes}
        \item[*] 通常情况下${\{t\}}$表示第$t$个batch。
    \end{tablenotes}
    %%
    \end{threeparttable}
\end{table}

%%%%
\subsection{Train/Dev/Test sets}
对于数据集,我们通常将其分为三个部分:
\begin{itemize}
    \item \textbf{Training set}: 训练集,用于训练模型。一般占60\%。
    \item \textbf{Dev (development) set}: 开发集,用于调整模型的超参数。一般占20\%。
    \item \textbf{Test set}: 测试集,用于测试模型的性能。一般占20\%。
\end{itemize}
有时候,我们会将数据集仅分为训练集和开发集,而没有测试集。这时候,我们通常将二者的比例调整为70\%和30\%。

%%%%
\subsection{Bias and Variance}
\textbf{偏差(Bias)}描述了模型的预测值与真实值之间的差距。Bias越大,说明模型越不准确。
而\textbf{方差(Variance)}描述了模型的预测值与真实值之间的差距的方差。Variance越大,说明模型越不稳定。

\begin{figure}[h!bt]
    \centering
    \subfigure[High Bias]{
        \begin{minipage}[t]{0.3\textwidth}
            \centering
            \includegraphics[scale=0.75]{high_bias.pdf}
        \end{minipage}
    }%
    \subfigure[Just Right]{
        \begin{minipage}[t]{0.3\textwidth}
            \centering
            \includegraphics[scale=0.75]{just_right.pdf}
        \end{minipage}
    }%
    \subfigure[High Variance]{
        \begin{minipage}[t]{0.3\textwidth}
            \centering
            \includegraphics[scale=0.75]{high_variance.pdf}
        \end{minipage}
    }%
    \centering
    \caption{Bias and Variance}
    \label{fig:bias-variance}
\end{figure}

\textbf{高偏差(High bias)}通常表示模型欠拟合,即模型的复杂度不够,无法拟合数据。这可以体现在模型在训练集上的误差和在开发集上的误差都很大,且二者之间的差距不大。
而\textbf{高方差(High variance)}通常表示模型过拟合,即模型的复杂度过高,导致模型过于敏感,无法泛化。这可以体现在模型在训练集上的误差很小,但在开发集上的误差很大。
若高偏差和高方差同时存在,则会体现在模型在训练集上的误差和在开发集上的误差都很大,且开发集上的误差明显大于训练集上的误差。

当我们发现模型的偏差很大时,可以尝试使用更大的网络、训练更长的时间、使用更好的优化算法等来提高模型的复杂度,从而减小偏差。
而当我们发现模型的方差很大时,可以尝试使用更多的数据、使用正则化、使用dropout等来减小模型的复杂度,从而减小方差。

%%%%
\subsection{Regularization}
\index{Regularization}

正则化(Regularization)是一种减小模型复杂度的方法,可以防止模型过拟合。
正则化的方法有很多,这里介绍两种常用的正则化方法:$L_2$正则化和Dropout。

\begin{hint}
    正则化的目的是改变(通常是缩小)在更新权重和偏置时的梯度值,其本身并不干预模型的前向传播和反向传播过程。
\end{hint}

\subsubsection{$\mathsf{L_2}$ regularization}

$L_2$正则化($L_2$ regularization)也称为权重衰减(Weight Decay),是一种常用的正则化方法。
$L_2$正则化通过在代价函数中添加一个权重
\footnote{一般不对偏置做$L_2$正则化,避免影响神经元的激活阈值。}
的$L_2$范数
\footnote{$L_2$范数是针对向量的,而Frobenius范数是针对矩阵的,在这里并没有做严谨的区分。}
惩罚项来实现,这个惩罚项模型权重的平方成正比。这样做的目的是鼓励模型使用较小的权重,从而减少模型的复杂度。

$L_2$范数的定义是,对于一个长度为$n$的向量$w$,其$L_2$范数为:
\begin{equation}
    ||w||_2 = \sqrt{\sum_{j=1}^nw_j^2}
\end{equation}

使用$L_2$正则化后,代价函数中添加了各层权重信息的Frobenius范数,其计算公式为:
\begin{equation}
    J = \frac{1}{m}\sum_{i=1}^mL(\hat{y}^{(i)}, y^{(i)}) + \frac{\lambda}{2m}\sum_{l=1}^L||W^{[l]}||_{\mathrm{F}}^2
\end{equation}
其中$\lambda$为正则化参数,用于控制正则化的强度。$||W^{[l]}||_2^2$表示第$l$层权重Frobenius范数的平方,即
\begin{equation}
    ||W^{[l]}||_{\mathrm{F}}^2 = \sum_{i=1}^{n^{[l]}}\sum_{j=1}^{n^{[l-1]}}(W_{ij}^{[l]})^2
\end{equation}
添加这个惩罚项后,代价函数对权重的梯度为:
\begin{equation}
    \frac{\d J}{\d W^{[l]}} = \frac{\d J}{\d Z^{[l]}} A^{[l-1] \mathrm{T}} + \frac{\lambda}{m}W^{[l]}
\end{equation}
其他的梯度表达式保持不变(证明略)。

\subsubsection{Dropout}
\index{Dropout}

Dropout是一种常用的正则化方法,可以防止模型过拟合。Dropout的基本思想是,在每次迭代中,随机地关闭一些神经元,从而减小模型的复杂度,相当于只训练了完整模型的一个子集。
Dropout 是以层为单位来使用的,即在设定的某些 Dropout 层中,每次迭代时,以一定的概率关闭其中的神经元(每次迭代可以关闭不同的神经元)。

需要注意,“关闭”神经元并不是将神经元中的权重和偏置清零,而只是将其输出置零。

\begin{figure}[h!bt]
    \centering
    \subfigure[随机选取神经元]{
        \begin{minipage}[t]{0.48\textwidth}
            \centering
            \includegraphics[width=0.8\textwidth]{dropout_1.pdf}
        \end{minipage}
    }%
    \subfigure[输出置零]{
        \begin{minipage}[t]{0.48\textwidth}
            \centering
            \includegraphics[width=0.8\textwidth]{dropout_2.pdf}
        \end{minipage}
    }%
    \centering
    \caption{Dropout}
    \label{fig:dropout}
\end{figure}

一种常见的Dropout实现为Inverted dropout。假设第$l$层被设置为Dropout层,设置一个保留概率 \verb|keep_prob| ,
则在每次迭代的前向传播中,先用原有的方式计算出第$l$层的激活值$A^{[l]}$,然后构造一个布尔矩阵(mask)$D^{[l]}$,其中的元素以概率 \verb|keep_prob| 取值为1,以概率 \verb|1-keep_prob| 取值为0。
\footnote{严格来说,“关闭”某些神经元应当将$A^{[l]} \in \mathbb{R}^{n^{[l]} \times {m}}$的某些行随机置零,而不是对所有元素随机置零。对所有元素随机置零相当于在同一轮迭代中每个样本都关闭了不同的神经元,但这样做也是可行的。}
\begin{equation}
    \mathtt{D^{\textcolor{Magenta}{[}l\textcolor{Magenta}{]}}} = \mathtt{np.random.rand\textcolor{yellow}{(} A^{\textcolor{Magenta}{[}l\textcolor{Magenta}{]}}.shape\textcolor{Magenta}{[}\textcolor{green}{0}\textcolor{Magenta}{]}, A^{\textcolor{Magenta}{[}l\textcolor{Magenta}{]}}.shape\textcolor{Magenta}{[}\textcolor{green}{1}\textcolor{Magenta}{]} \textcolor{yellow}{)} < keep\_prob}
\end{equation}
将这个布尔矩阵与$A^{[l]}$相乘,即可将第$l$层的部分输出置零
\begin{equation}
    \mathtt{A^{\textcolor{Magenta}{[}l\textcolor{Magenta}{]}}} = \mathtt{np.multiply\textcolor{yellow}{(} A^{\textcolor{Magenta}{[}l\textcolor{Magenta}{]}}, D^{\textcolor{Magenta}{[}l\textcolor{Magenta}{]}} \textcolor{yellow}{)}}
\end{equation}
为了补偿训练过程中由于 Dropout 导致的输出值的减少,需要进行一次缩放,即
\begin{equation}
    \mathtt{A^{\textcolor{Magenta}{[}l\textcolor{Magenta}{]}}} /= \mathtt{keep\_prob}
\end{equation}
将这个值作为第$l$层的输出$A^{[l]}$。这个操作应在计算第$l+1$层之前完成。

在反向传播中,使用这些经过 Dropout 的各层输出来计算梯度来更新权重和偏置。同时,$\d A^{[l]}$也需要进行类似的操作使其与$A^{[l]}$对应,即
\begin{align}
    \mathtt{dA^{\textcolor{Magenta}{[}l\textcolor{Magenta}{]}}} &= \mathtt{np.multiply\textcolor{yellow}{(} dA^{\textcolor{Magenta}{[}l\textcolor{Magenta}{]}}, D^{\textcolor{Magenta}{[}l\textcolor{Magenta}{]}} \textcolor{yellow}{)}} \\
    \mathtt{dA^{\textcolor{Magenta}{[}l\textcolor{Magenta}{]}}} &/= \mathtt{keep\_prob}
\end{align}
这个操作应在计算第$l-1$层之前完成。

从结果上看,Dropout 操作相当于这一轮迭代中的训练的神经元数“减少”了,这可以降低模型对单个神经元的依赖,从而提高模型的泛化能力。

在评估模型或者使用模型进行预测时(此时只有前向传播),需要关闭Dropout。这是为了确保模型的输出与训练阶段保持一致,从而准确评估模型在未见数据上的性能。

\subsubsection{Other regularization methods}

除了$L_2$正则化和Dropout,还有一些其他的正则化方法,如数据增强(Data Augmentation)、早停(Early Stopping)等。

数据增强是一种简单有效的正则化方法,其基本思想是通过对训练集中的数据进行一些随机变换,从而生成更多的训练数据。
例如,对于图像分类问题,可以对图像进行一些随机的旋转、平移、缩放等操作,将其作为新增的训练数据。

早停的基本思想是在训练过程中,当模型在开发集上的误差连续一段时间不再下降时,就停止训练。
这样可以防止模型过拟合,但需要注意的是,早停的效果很大程度上依赖于开发集的选择。

\subsection{Normalizing Inputs}
\index{Normalization}

在训练神经网络时,我们通常会对输入数据进行归一化处理。
归一化可以确保数值在可接受的范围内,将所有特征重现在相同的尺度上,提高算法的数值稳定性。

数据归一化通常包括两个步骤:减去平均值(使得归一化后的数据具有零均值)和除以标准差(使得归一化后的数据具有单位方差)。
这种方法也被称为 Z-score 标准化或标准化。在标准化后,数据的均值为0,标准差为1。

\begin{figure}[h!bt]
    \centering
    \subfigure[Before Normalization]{
        \begin{minipage}[t]{0.48\textwidth}
            \centering
            \includegraphics[scale=0.75]{norm_before.pdf}
        \end{minipage}
    }%
    \subfigure[After Normalization]{
        \begin{minipage}[t]{0.48\textwidth}
            \centering
            \includegraphics[scale=0.75]{norm_after.pdf}
        \end{minipage}
    }%
    \centering
    \caption{Normalization}
    \label{fig:normalization}
\end{figure}

在 python 中,可以很方便地实现输入数据的归一化
\begin{python}
X -= np.mean(X, axis=0)
X /= np.std(X, axis=0)
\end{python}

%%%
\section{Optimization Algorithms}

%%%%
\subsection{Mini-batch Gradient Descent}
\index{Mini-batch Gradient Descent}

之前的梯度下降算法都是基于整个训练集的,即每次迭代都要计算整个训练集的代价函数和梯度,然后更新权重和偏置。
这样的梯度下降算法称为\textbf{批量梯度下降(Batch Gradient Descent)},其缺点是每次更新梯度都需要完整遍历整个训练集,计算量很大,效率很低。
一种更好的方法是将训练集分成若干个子集,每次迭代只计算一个子集的代价函数和梯度,然后更新权重和偏置。
这样的梯度下降算法称为\textbf{小批量梯度下降(Mini-batch Gradient Descent)}。

\begin{hint}
    权重$W$和偏置$b$的规模完全由各层节点数$n^{[l]},\; 0 \leqslant l \leqslant L$决定,与样本数$m$无关。对于不同的样本数$m$,只需要调整$X$和$Y$的规模即可,而网络内部的节点数不需要调整就能适应不同规模的样本。
    这也是 Mini-batch 优化算法能实现的基础。
\end{hint}

在实现时,设定一个 \verb|mini_batch_size| (可简写为 \verb|batch_size|) ,将训练集打乱顺序,然后将其分成若干个大小为 \verb|batch_size| 的子集。
对于不同的 \verb|mini_batch| ,仍共享相同的权重和偏置,即训练的仍是同一个网络。

前向传播的过程与之前的算法相同,相当于将第$t$个子集$X^{\{t\}}$作为输入矩阵,依次计算出各层的输出矩阵$A^{[l]}$。
应当注意,此时输入的样本的规模相当于从$m$变成了 \verb|batch_size|,在计算代价函数时需要除以 \verb|batch_size| 而不是$m$。例如在使用 $L_2$ 正则化时,代价函数的计算公式为
\begin{equation}
    J^{\{t\}} = \frac{1}{\mathtt{batch\_size}}\sum_{i=1}^\mathtt{batch\_size} L(\hat{y}^{(i)}, y^{(i)}) + \frac{\lambda}{2\;\mathtt{batch\_size}}\sum_{l=1}^L||W^{[l]}||_{\mathrm{F}}^2
\end{equation}
不同的 \verb|mini_batch| 包含不同的样本,因此即使使用同样的权重和偏置,不同的 \verb|mini_batch| 也会得到不同的代价函数。
在使用小批量梯度下降时,代价函数并不是像之前那样单调递减的,而是有一定的波动。

\begin{figure}[h!bt]
    \centering
    \includegraphics[scale=0.65]{mini-batch_cost.pdf}
    \caption{Cost Function in Mini-batch Gradient Descent}
    \label{fig:mini-batch_cost}
\end{figure}

反向传播的过程也与之前的算法相同,依次计算出各层的梯度矩阵$\d W^{[l]}$和$\d b^{[l]}$,然后更新权重和偏置。

当所有的 \verb|mini_batch| 都遍历完一遍训练集后,称为完成了一个 \texttt{\textbf{epoch}}。
训练过程中使用的 \verb|epoch| 数也是一个超参数,表示了整个训练集被遍历的次数。在每个 \verb|epoch| 结束后,可以重新打乱训练集的顺序,然后再次分成若干个 \verb|mini_batch| 继续训练以增强随机性。

当数据量非常大时,使用小批量梯度下降算法已被证明是一种有效的优化算法,可以加快训练速度。
小批量梯度下降算法每次更新权重和偏置只需要计算一个 \verb|mini_batch| 的代价函数和梯度,相比于批量梯度下降算法大大减少了计算量,可以更快地收敛。

\verb|batch_size| 的取值一般为2的幂次方,如32、64、128等,一般不会超过512。过大的 \verb|batch_size| 会导致训练速度变慢,而过小的 \verb|batch_size| 会导致训练过程中的代价函数波动较大。

%%%%
\subsection{Gradient Descent with Momentum}

在讨论矩估计时,符号标记忽略了层数$l$便于书写,但在实际实现时,应当使用$l$来区分不同层的矩估计。

动量可以平均梯度的方向,从而减少梯度下降的震荡。

一般不删除系数 $1 - \beta$ 

	这是一篇关于 LaTeX 的参考文献 \cite{lamport1994latex}。

	% % % % % % % % % %
	\appendix
\chapter{Mathmatical}

%%%%
\section{Matrix derivative}
\label{sec:matrix_derivative}
\index{Matrix derivative}

\begin{wenxintishi}
	在本节中,为便于区分,用粗体大写字母表示矩阵,用粗体小写字母表示向量,用普通小写字母表示标量。这些表示与正文不同。
\end{wenxintishi}

矩阵微积分的表示通常有两种符号约定,分别是\textbf{分子布局(Numerator Layout)}和\textbf{分母布局(Denominator Layout)}\cite{matrix_cookbook}。下面将分别介绍这两种符号约定。

%%%
\subsection{标量关于向量的偏导数}
如果一个标量函数$f(\bm{x})$的自变量是一个$n$维向量$\bm{x}$,那么这个标量函数就可以对这个向量$\bm{x}$求偏导数。

在分子布局下,这个偏导数是一个行向量,其第$i$个元素是$f(\bm{x})$对$\bm{x}$的第$i$个元素的偏导数,即
\begin{equation}
	\frac{\partial f(\bm{x})}{\partial \bm{x}}
	=\left[\frac{\partial f(\bm{x})}{\partial x_1},\frac{\partial f(\bm{x})}{\partial x_2},\cdots,\frac{\partial f(\bm{x})}{\partial x_n}\right]_{\substack{\scriptstyle 1\times n}}
\end{equation}
而在分母布局下,这个偏导数是一个列向量,其第$i$个元素是$f(\bm{x})$对$\bm{x}$的第$i$个元素的偏导数,即
\begin{equation}
	\frac{\partial f(\bm{x})}{\partial \bm{x}}
	=\begin{bmatrix}
		\dfrac{\partial f(\bm{x})}{\partial x_1}\\[2ex]
		\dfrac{\partial f(\bm{x})}{\partial x_2}\\[2ex]
		\vdots\\[2ex]
		\dfrac{\partial f(\bm{x})}{\partial x_n}
	\end{bmatrix}_{\substack{\scriptstyle n\times 1}}
\end{equation}

%%%
\subsection{向量关于标量的偏导数}
如果一个$m$维向量函数$\bm{f}(x)$的自变量是一个标量$x$,那么这个向量函数就可以对这个标量$x$求偏导数。


在分子布局下,这个偏导数是一个列向量,其第$i$个元素是$\bm{f}(x)$对$x$的偏导数,即
\begin{equation}
	\frac{\partial \bm{f}(x)}{\partial x}
	=\begin{bmatrix}
		\dfrac{\partial f_1(x)}{\partial x}\\[2ex]
		\dfrac{\partial f_2(x)}{\partial x}\\[2ex]
		\vdots\\[2ex]
		\dfrac{\partial f_m(x)}{\partial x}
	\end{bmatrix}_{\substack{\scriptstyle m\times 1}}
\end{equation}
而在分母布局下,这个偏导数是一个行向量,其第$i$个元素是$\bm{f}(x)$对$x$的偏导数,即
\begin{equation}
	\frac{\partial \bm{f}(x)}{\partial x}
	=\left[\frac{\partial f_1(x)}{\partial x},\frac{\partial f_2(x)}{\partial x},\cdots,\frac{\partial f_m(x)}{\partial x}\right]_{\substack{\scriptstyle 1\times m}}
\end{equation}

%%%
\subsection{向量关于向量的偏导数}
如果一个$m$维向量函数$\bm{f}(\bm{x})$的自变量是一个$n$维向量$\bm{x}$,那么这个向量函数就可以对这个向量$\bm{x}$求偏导数。

在分子布局下,这个偏导数是一个$m\times n$维矩阵,其第$i$行第$j$列的元素是$f_i(\bm{x})$对$\bm{x}$的第$j$个元素的偏导数,即
\begin{equation}
	\frac{\partial \bm{f}(\bm{x})}{\partial \bm{x}}
	=\left[\frac{\partial \bm{f}(\bm{x})}{\partial x_1},\frac{\partial \bm{f}(\bm{x})}{\partial x_2},\cdots,\frac{\partial \bm{f}(\bm{x})}{\partial x_n}\right]
	=\begin{bmatrix}
		\dfrac{\partial f_1(\bm{x})}{\partial x_1}&\dfrac{\partial f_1(\bm{x})}{\partial x_2}&\cdots&\dfrac{\partial f_1(\bm{x})}{\partial x_n}\\[2ex]
		\dfrac{\partial f_2(\bm{x})}{\partial x_1}&\dfrac{\partial f_2(\bm{x})}{\partial x_2}&\cdots&\dfrac{\partial f_2(\bm{x})}{\partial x_n}\\[2ex]
		\vdots&\vdots&\ddots&\vdots\\[2ex]
		\dfrac{\partial f_m(\bm{x})}{\partial x_1}&\dfrac{\partial f_m(\bm{x})}{\partial x_2}&\cdots&\dfrac{\partial f_m(\bm{x})}{\partial x_n}
	\end{bmatrix}_{\substack{\scriptstyle m\times n}}
\end{equation}
而在分母布局下,这个偏导数是一个$n\times m$维矩阵,其第$i$行第$j$列的元素是$f_j(\bm{x})$对$\bm{x}$的第$i$个元素的偏导数,即
\begin{equation}
	\frac{\partial \bm{f}(\bm{x})}{\partial \bm{x}}
	=\left[\frac{\partial f_1(\bm{x})}{\partial \bm{x}},\frac{\partial f_2(\bm{x})}{\partial \bm{x}},\cdots,\frac{\partial f_m(\bm{x})}{\partial \bm{x}}\right]
	=\begin{bmatrix}
		\dfrac{\partial f_1(\bm{x})}{\partial x_1}&\dfrac{\partial f_2(\bm{x})}{\partial x_1}&\cdots&\dfrac{\partial f_m(\bm{x})}{\partial x_1}\\[2ex]
		\dfrac{\partial f_1(\bm{x})}{\partial x_2}&\dfrac{\partial f_2(\bm{x})}{\partial x_2}&\cdots&\dfrac{\partial f_m(\bm{x})}{\partial x_2}\\[2ex]
		\vdots&\vdots&\ddots&\vdots\\[2ex]
		\dfrac{\partial f_1(\bm{x})}{\partial x_n}&\dfrac{\partial f_2(\bm{x})}{\partial x_n}&\cdots&\dfrac{\partial f_m(\bm{x})}{\partial x_n}
	\end{bmatrix}_{\substack{\scriptstyle n\times m}}
\end{equation}

一个小结论是,当使用分子布局时,将分子看作列向量,结果的行数与分子相同;而当使用分母布局时,将分母看作列向量,结果的行数与分母相同。

%%%
\subsection{标量关于矩阵的偏导数}
\footnote{本节内容参考了\cite{matrix_derivative}}

在深度学习中,代价函数是一个标量函数,而神经网络的参数通常是一个矩阵。因此,需要求标量函数对矩阵的偏导数。

标量对矩阵求导,可以看作是标量对矩阵的每个元素求导,然后将结果组合成一个矩阵。结果的维度应当与被求导矩阵的维度相同,便于进行梯度下降,故应采用分母布局。本节的内容从这里开始均采用\textcolor{red}{分母布局}。

一元微积分中的导数(标量对标量的导数)满足
\begin{equation}
	\mathrm{d}f 
	= \frac{\partial f}{\partial x}\mathrm{d}x
	= \langle \frac{\partial f}{\partial x}, \mathrm{d}x \rangle
\end{equation}
其中$\langle \bm{u}, \bm{v} \rangle$表示$\bm{u}$与$\bm{v}$的内积。
在多元微积分(标量对向量的导数)中,这个关系可以推广为
\begin{equation}
	\mathrm{d}f
	= \sum_{i=1}^{n}{\frac{\partial f}{\partial x_i}\mathrm{d}x_i} 
	= {\frac{\partial f}{\partial \bm{x}}}^{\mathrm{T}}\mathrm{d}\bm{x}
	= \langle \frac{\partial f}{\partial \bm{x}}, \mathrm{d}\bm{x} \rangle
\end{equation}
类似地,标量对矩阵的导数与微分的关系可以推广为
\begin{equation}
	\mathrm{d}f
	= \langle \frac{\partial f}{\partial \bm{X}}, \mathrm{d}\bm{X} \rangle
	\label{eq:scalar_matrix_derivative}
\end{equation}
两个大小相同矩阵的内积可以表示为
\begin{equation}
	\langle \bm{A}, \bm{B} \rangle 
	= \sum_{i=1}^{m}{\sum_{j=1}^{n}{\frac{\partial f}{\partial x_{ij}}\mathrm{d}x_{ij}}} 
	= \mathrm{tr}(\bm{A}^{\mathrm{T}}\bm{B})
\end{equation}
故式\eqref{eq:scalar_matrix_derivative}可以表示为
\begin{equation}
	\eqnmarkbox[WildStrawberry]{scalar_matrix_derivative_trace}{
	\mathrm{d}f 
	= \mathrm{tr}\left({\frac{\partial f}{\partial \bm{X}}}^{\mathrm{T}}\mathrm{d}\bm{X}\right)
	}
	\label{eq:scalar_matrix_derivative_trace}
\end{equation}
这便是标量对矩阵求导与微分的关系。只要满足式\eqref{eq:scalar_matrix_derivative_trace}的关系,就能得到标量对矩阵的导数。

\vspace{0.5\baselineskip}
下面列出几条常见的矩阵微分的运算法则。注意$\odot$表示Hadamard积,即对应元素相乘。
\begin{subequations}
	\begin{align}
		\mathrm{d}(\bm{X}\pm \bm{Y}) &= \mathrm{d}\bm{X} \pm \mathrm{d}\bm{Y} 
		\label{eq:matrix_derivative_add_sub} \\
		\mathrm{d}(\bm{XY}) &= \mathrm{d}\bm{X}\cdot \bm{Y} + \bm{X}\cdot \mathrm{d}\bm{Y}
		\label{eq:matrix_derivative_mul} \\
		\mathrm{d}(\bm{X}^{\mathrm{T}}) &= (\mathrm{d}\bm{X})^{\mathrm{T}}
		\label{eq:matrix_derivative_transpose} \\
		\mathrm{d}(\bm{X}^{-1}) &= -\bm{X}^{-1}\mathrm{d}\bm{X}\bm{X}^{-1}
		\label{eq:matrix_derivative_inverse} \\
		\mathrm{d}(\bm{X}\odot \bm{Y}) &= \mathrm{d}\bm{X}\odot \bm{Y} + \bm{X}\odot \mathrm{d}\bm{Y}
		\label{eq:matrix_derivative_hadamard} \\
		\mathrm{d}\left(g(\bm{X})\right) &= g'(\bm{X}) \odot \mathrm{d}\bm{X}
		\label{eq:matrix_derivative_hadamard2}
	\end{align}
\end{subequations}

对于矩阵的迹(trace)运算,有以下常用的性质,在后面的推导中会用到。
\begin{subequations}
	\begin{align}
		a &= \mathrm{tr}(a)
		\label{eq:matrix_trace_scalar} \\
		\mathrm{tr}(\bm{A}^{\mathrm{T}}) &= \mathrm{tr}(\bm{A})
		\label{eq:matrix_trace_transpose} \\
		\mathrm{tr}(\bm{A}\pm\bm{B}) &= \mathrm{tr}(\bm{A}) \pm \mathrm{tr}(\bm{B})
		\label{eq:matrix_trace_add_sub} \\
		\mathrm{tr}(\bm{AB}) &= \mathrm{tr}(\bm{BA})
		\label{eq:matrix_trace_mul} \\
		\mathrm{tr}\left(\bm{A}^{\mathrm{T}}(\bm{B} \odot \bm{C})\right) &= \mathrm{tr}\left((\bm{A} \odot \bm{B})^{\mathrm{T}}\bm{C}\right)
		\label{eq:matrix_trace_hadamard}
	\end{align}
\end{subequations}

在复合情况下,为了避免矩阵对矩阵求导,尽量不要使用链式法则,而是使用微分的性质\eqref{eq:scalar_matrix_derivative_trace},通过迹运算的变形来求解。

\begin{example}\label{example:matrix_derivative1}
\vspace{0.5\baselineskip}
\colorbox{LemonChiffon}{已知$\dfrac{\partial f}{\partial \bm{Y}}$和$\bm{Y}=\bm{A}\bm{X}\bm{B}$,求$\dfrac{\partial f}{\partial \bm{X}}$}。
根据式\eqref{eq:scalar_matrix_derivative_trace},有
\begin{equation}
	\mathrm{d}f = \mathrm{tr}\left({\frac{\partial f}{\partial \bm{Y}}}^{\mathrm{T}}\mathrm{d}\bm{Y}\right)
	\label{eq:matrix_derivative_example1_1}
\end{equation}
对式\eqref{eq:matrix_derivative_example1_1}中的$\mathrm{d}\bm{Y}$进行展开,有
\begin{equation}
	\mathrm{d}\bm{Y} = \mathrm{d}(\bm{A}\bm{X}\bm{B}) = \mathrm{d}\bm{A}\cdot\bm{X}\bm{B} + \bm{A}\mathrm{d}\bm{X}\cdot\bm{B} + \bm{A}\bm{X}\mathrm{d}\bm{B}
	\label{eq:matrix_derivative_example1_2}
\end{equation}
在$A$, $B$均为常量的情况下,式\eqref{eq:matrix_derivative_example1_2}中的第一项和第三项为零,故式\eqref{eq:matrix_derivative_example1_1}可以化简为
\begin{equation}
	\mathrm{d}f = \mathrm{tr}\left({\frac{\partial f}{\partial \bm{Y}}}^{\mathrm{T}}\bm{A}\mathrm{d}\bm{X}\bm{B}\right)
	\label{eq:matrix_derivative_example1_3}
\end{equation}
根据式\eqref{eq:matrix_trace_mul},式\eqref{eq:matrix_derivative_example1_3}可以进一步改写为
\renewcommand{\eqnhighlightheight}{\vphantom{{\frac{\partial f}{\partial \bm{Y}}}^{\mathrm{T}}}\mathstrut}  % 调整注释框高度便于统一,默认为空
\begin{equation}
	\mathrm{d}f 
	= \mathrm{tr}\left(\eqnmarkbox[blue]{c1}{{\frac{\partial f}{\partial \bm{Y}}}^{\mathrm{T}}\bm{A}\mathrm{d}\bm{X}} \eqnmarkbox[red]{c2}{\bm{B}}\right)
	= \mathrm{tr}\left(\eqnmark[RoyalPurple]{t1}{\bm{B}{\frac{\partial f}{\partial \bm{Y}}}^{\mathrm{T}}\bm{A}} \mathrm{d}\bm{X}\right)
	= \mathrm{tr}\left(\eqnmark[RoyalPurple]{t2}{\left(\bm{A}^{\mathrm{T}}\frac{\partial f}{\partial \bm{Y}}\bm{B}^{\mathrm{T}}\right)^{\mathrm{T}}} \mathrm{d}\bm{X}\right)
	\vspace{1em}
	\label{eq:matrix_derivative_example1_4}
	\tikzset{annotate equations/arrow/.style={color=ForestGreen, >=latex', semithick, dashed}}  % 注释箭头样式,双向绿色虚线箭头,默认为空
	\annotatetwo[yshift=-1em]{below, label below}{c1}{c2}{exchange}
	\tikzset{annotate equations/arrow/.style={->, semithick, dashed}}  % 单向箭头
	\annotatetwo[yshift=-1em]{below, label below}{t1}{t2}{equivalent}
\end{equation}
进而对比式\eqref{eq:scalar_matrix_derivative_trace},可得出
\renewcommand{\eqnhighlightshade}{100}  % 注释框不透明度,默认17
\begin{equation}
	\eqnmarkbox[LemonChiffon]{matrix_derivative_example1}{
	\frac{\partial f}{\partial \bm{X}} = \bm{A}^{\mathrm{T}}\frac{\partial f}{\partial \bm{Y}}\bm{B}^{\mathrm{T}}
	}
	\label{eq:matrix_derivative_example1_5}
\end{equation}
\end{example}

\begin{example}\label{example:matrix_derivative2}
\colorbox{Aquamarine!30}{已知$\dfrac{\partial f}{\partial \bm{Y}}$和$\bm{Y}=g(\bm{X})\odot\bm{X}$,求$\dfrac{\partial f}{\partial \bm{X}}$}。
我们同样有
\begin{equation}
	\mathrm{d}f = \mathrm{tr}\left({\frac{\partial f}{\partial \bm{Y}}}^{\mathrm{T}}\mathrm{d}\bm{Y}\right)
	\label{eq:matrix_derivative_example2_1}
\end{equation}
由式\eqref{eq:matrix_derivative_hadamard2},
\begin{equation}
	\mathrm{d}\bm{Y} = \mathrm{d}\left(g(\bm{X})\right) = g'(\bm{X}) \odot \mathrm{d}\bm{X}
	\label{eq:matrix_derivative_example2_2}
\end{equation}
代入\eqref{eq:matrix_derivative_example2_1},
\begin{equation}
	\mathrm{d}f = \mathrm{tr}\left({\frac{\partial f}{\partial \bm{Y}}}^{\mathrm{T}}\left(\vphantom{A^T}g'(\bm{X}) \odot \mathrm{d}\bm{X}\right)\right)
	\label{eq:matrix_derivative_example2_3}
\end{equation}
根据式\eqref{eq:matrix_trace_hadamard},式\eqref{eq:matrix_derivative_example2_3}可以进一步改写为
\begin{equation}
	\begin{aligned}
		\mathrm{d}f 
		&= \mathrm{tr}\left(\eqnmark[Olive]{}{\frac{\partial f}{\partial \bm{Y}}}^{\mathrm{T}}\left(\vphantom{A^T}\eqnmark[Teal]{}{g'(\bm{X})} \odot \eqnmark[Maroon]{}{\mathrm{d}\bm{X}}\right)\right) \\
		&= \mathrm{tr}\left(\left({\eqnmark[Olive]{}{\frac{\partial f}{\partial \bm{Y}}}\odot \eqnmark[Teal]{}{g'(\bm{X})}}\right)^{\mathrm{T}}\eqnmark[Maroon]{}{\mathrm{d}\bm{X}}\right) \\
		&= \mathrm{tr}\left(\left({\eqnmark[Teal]{}{g'(\bm{X})} \odot \eqnmark[Olive]{}{\frac{\partial f}{\partial \bm{Y}}}}\right)^{\mathrm{T}}\eqnmark[Maroon]{}{\mathrm{d}\bm{X}}\right)
	\end{aligned}
\end{equation}
对比式\eqref{eq:scalar_matrix_derivative_trace},可得出
\renewcommand{\eqnhighlightshade}{30}  % 注释框不透明度,默认17
\begin{equation}
	\eqnmarkbox[Aquamarine]{matrix_derivative_example2}{
	\frac{\partial f}{\partial \bm{X}} = g'(\bm{X}) \odot \frac{\partial f}{\partial \bm{Y}}
	}
	\label{eq:matrix_derivative_example2_4}
\end{equation}
\end{example}

\vspace{0.5\baselineskip}
从例\ref{example:matrix_derivative1}和例\ref{example:matrix_derivative2}可得到下面的结论
\begin{align}
	\bm{Y}=\bm{A}\bm{X}\bm{B} \quad &\Rightarrow \quad \frac{\partial f}{\partial \bm{X}} = \bm{A}^{\mathrm{T}}\frac{\partial f}{\partial \bm{Y}}\bm{B}^{\mathrm{T}}
	\label{eq:matrix_derivative_conclusion1} \\ %%
	\bm{Y}=g(\bm{X})\odot\bm{X} \quad &\Rightarrow \quad \frac{\partial f}{\partial \bm{X}} = g'(\bm{X}) \odot \frac{\partial f}{\partial \bm{Y}} \vphantom{{\frac{\partial f}{\partial \bm{X}}}^{\mathrm{T}}}
	\label{eq:matrix_derivative_conclusion2}  %%
\end{align}


	% % % % % % % % % %
	\backmatter
	% 使用 bibLaTeX 制作书目
	\printbibliography[heading=bibintoc]
	
	% 图, 表索引. 可有可无.
	\listoffigures
	\listoftables

	% 制作索引 (用 imakeidx 的功能可以制作多份)
	{\footnotesize
	\indexprologue{中文术语按汉语拼音排序.}
	\printindex}

\end{document}
