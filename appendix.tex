\appendix
\chapter{Mathmatical}

%%%%
\section{Matrix derivative}
\label{sec:matrix_derivative}
\index{Matrix derivative}

\begin{hint}
	在本节中,为便于区分,用粗体大写字母表示矩阵,用粗体小写字母表示向量,用普通小写字母表示标量。这些表示与正文不同。
\end{hint}

矩阵微积分的表示通常有两种符号约定,分别是\textbf{分子布局(Numerator Layout)}和\textbf{分母布局(Denominator Layout)}(可参考\cite{petersen2008matrix})。下面将分别介绍这两种符号约定。

%%%
\subsection{标量关于向量的偏导数}
如果一个标量函数$f(\bm{x})$的自变量是一个$n$维向量$\bm{x}$,那么这个标量函数就可以对这个向量$\bm{x}$求偏导数。

在分子布局下,这个偏导数是一个行向量,其第$i$个元素是$f(\bm{x})$对$\bm{x}$的第$i$个元素的偏导数,即
\begin{equation}
	\frac{\partial f(\bm{x})}{\partial \bm{x}}
	=\left[\frac{\partial f(\bm{x})}{\partial x_1},\frac{\partial f(\bm{x})}{\partial x_2},\cdots,\frac{\partial f(\bm{x})}{\partial x_n}\right]_{\substack{\scriptstyle 1\times n}}
\end{equation}
而在分母布局下,这个偏导数是一个列向量,其第$i$个元素是$f(\bm{x})$对$\bm{x}$的第$i$个元素的偏导数,即
\begin{equation}
	\frac{\partial f(\bm{x})}{\partial \bm{x}}
	=\begin{bmatrix}
		\dfrac{\partial f(\bm{x})}{\partial x_1}\\[2ex]
		\dfrac{\partial f(\bm{x})}{\partial x_2}\\[2ex]
		\vdots\\[2ex]
		\dfrac{\partial f(\bm{x})}{\partial x_n}
	\end{bmatrix}_{\substack{\scriptstyle n\times 1}}
\end{equation}

%%%
\subsection{向量关于标量的偏导数}
如果一个$m$维向量函数$\bm{f}(x)$的自变量是一个标量$x$,那么这个向量函数就可以对这个标量$x$求偏导数。


在分子布局下,这个偏导数是一个列向量,其第$i$个元素是$\bm{f}(x)$对$x$的偏导数,即
\begin{equation}
	\frac{\partial \bm{f}(x)}{\partial x}
	=\begin{bmatrix}
		\dfrac{\partial f_1(x)}{\partial x}\\[2ex]
		\dfrac{\partial f_2(x)}{\partial x}\\[2ex]
		\vdots\\[2ex]
		\dfrac{\partial f_m(x)}{\partial x}
	\end{bmatrix}_{\substack{\scriptstyle m\times 1}}
\end{equation}
而在分母布局下,这个偏导数是一个行向量,其第$i$个元素是$\bm{f}(x)$对$x$的偏导数,即
\begin{equation}
	\frac{\partial \bm{f}(x)}{\partial x}
	=\left[\frac{\partial f_1(x)}{\partial x},\frac{\partial f_2(x)}{\partial x},\cdots,\frac{\partial f_m(x)}{\partial x}\right]_{\substack{\scriptstyle 1\times m}}
\end{equation}

%%%
\subsection{向量关于向量的偏导数}
如果一个$m$维向量函数$\bm{f}(\bm{x})$的自变量是一个$n$维向量$\bm{x}$,那么这个向量函数就可以对这个向量$\bm{x}$求偏导数。

在分子布局下,这个偏导数是一个$m\times n$维矩阵,其第$i$行第$j$列的元素是$f_i(\bm{x})$对$\bm{x}$的第$j$个元素的偏导数,即
\begin{equation}
	\frac{\partial \bm{f}(\bm{x})}{\partial \bm{x}}
	=\left[\frac{\partial \bm{f}(\bm{x})}{\partial x_1},\frac{\partial \bm{f}(\bm{x})}{\partial x_2},\cdots,\frac{\partial \bm{f}(\bm{x})}{\partial x_n}\right]
	=\begin{bmatrix}
		\dfrac{\partial f_1(\bm{x})}{\partial x_1}&\dfrac{\partial f_1(\bm{x})}{\partial x_2}&\cdots&\dfrac{\partial f_1(\bm{x})}{\partial x_n}\\[2ex]
		\dfrac{\partial f_2(\bm{x})}{\partial x_1}&\dfrac{\partial f_2(\bm{x})}{\partial x_2}&\cdots&\dfrac{\partial f_2(\bm{x})}{\partial x_n}\\[2ex]
		\vdots&\vdots&\ddots&\vdots\\[2ex]
		\dfrac{\partial f_m(\bm{x})}{\partial x_1}&\dfrac{\partial f_m(\bm{x})}{\partial x_2}&\cdots&\dfrac{\partial f_m(\bm{x})}{\partial x_n}
	\end{bmatrix}_{\substack{\scriptstyle m\times n}}
\end{equation}
而在分母布局下,这个偏导数是一个$n\times m$维矩阵,其第$i$行第$j$列的元素是$f_j(\bm{x})$对$\bm{x}$的第$i$个元素的偏导数,即
\begin{equation}
	\frac{\partial \bm{f}(\bm{x})}{\partial \bm{x}}
	=\left[\frac{\partial f_1(\bm{x})}{\partial \bm{x}},\frac{\partial f_2(\bm{x})}{\partial \bm{x}},\cdots,\frac{\partial f_m(\bm{x})}{\partial \bm{x}}\right]
	=\begin{bmatrix}
		\dfrac{\partial f_1(\bm{x})}{\partial x_1}&\dfrac{\partial f_2(\bm{x})}{\partial x_1}&\cdots&\dfrac{\partial f_m(\bm{x})}{\partial x_1}\\[2ex]
		\dfrac{\partial f_1(\bm{x})}{\partial x_2}&\dfrac{\partial f_2(\bm{x})}{\partial x_2}&\cdots&\dfrac{\partial f_m(\bm{x})}{\partial x_2}\\[2ex]
		\vdots&\vdots&\ddots&\vdots\\[2ex]
		\dfrac{\partial f_1(\bm{x})}{\partial x_n}&\dfrac{\partial f_2(\bm{x})}{\partial x_n}&\cdots&\dfrac{\partial f_m(\bm{x})}{\partial x_n}
	\end{bmatrix}_{\substack{\scriptstyle n\times m}}
\end{equation}

一个小结论是,当使用分子布局时,将分子看作列向量,结果的行数与分子相同;而当使用分母布局时,将分母看作列向量,结果的行数与分母相同。

%%%
\subsection[标量关于矩阵的偏导数]{标量关于矩阵的偏导数\footnote{本节内容参考了\cite{zhihu2020matrix}}}

在深度学习中,代价函数是一个标量函数,而神经网络的参数通常是一个矩阵。因此,需要求标量函数对矩阵的偏导数。

标量对矩阵求导,可以看作是标量对矩阵的每个元素求导,然后将结果组合成一个矩阵。结果的维度应当与被求导矩阵的维度相同,便于进行梯度下降,故应采用分母布局。本节的内容从这里开始均采用\textcolor{red}{分母布局}。

一元微积分中的导数(标量对标量的导数)满足
\begin{equation}
	\d f 
	= \frac{\partial f}{\partial x}\d x
	= \langle \frac{\partial f}{\partial x}, \d x \rangle
\end{equation}
其中$\langle \bm{u}, \bm{v} \rangle$表示$\bm{u}$与$\bm{v}$的内积。
在多元微积分(标量对向量的导数)中,这个关系可以推广为
\begin{equation}
	\d f
	= \sum_{i=1}^{n}{\frac{\partial f}{\partial x_i}\d x_i} 
	= {\frac{\partial f}{\partial \bm{x}}}^{\mathrm{T}}\d \bm{x}
	= \langle \frac{\partial f}{\partial \bm{x}}, \d \bm{x} \rangle
\end{equation}
类似地,标量对矩阵的导数与微分的关系可以推广为
\begin{equation}
	\d f
	= \langle \frac{\partial f}{\partial \bm{X}}, \d \bm{X} \rangle
	\label{eq:scalar_matrix_derivative}
\end{equation}
两个大小相同矩阵的内积可以表示为
\begin{equation}
	\langle \bm{A}, \bm{B} \rangle 
	= \sum_{i=1}^{m}{\sum_{j=1}^{n}{\frac{\partial f}{\partial x_{ij}}\d x_{ij}}} 
	= \mathrm{tr}(\bm{A}^{\mathrm{T}}\bm{B})
\end{equation}
故式\eqref{eq:scalar_matrix_derivative}可以表示为
\begin{equation}
	\eqnmarkbox[WildStrawberry]{scalar_matrix_derivative_trace}{
	\d f 
	= \mathrm{tr}\left({\frac{\partial f}{\partial \bm{X}}}^{\mathrm{T}}\d \bm{X}\right)
	}
	\label{eq:scalar_matrix_derivative_trace}
\end{equation}
这便是标量对矩阵求导与微分的关系。只要满足式\eqref{eq:scalar_matrix_derivative_trace}的关系,就能得到标量对矩阵的导数。

下面列出几条常见的矩阵微分的运算法则。注意$\odot$表示Hadamard积,即对应元素相乘。
\begin{subequations}
	\begin{align}
		\d (\bm{X}\pm \bm{Y}) &= \d \bm{X} \pm \d \bm{Y} 
		\label{eq:matrix_derivative_add_sub} \\
		\d (\bm{XY}) &= \d \bm{X}\cdot \bm{Y} + \bm{X}\cdot \d \bm{Y}
		\label{eq:matrix_derivative_mul} \\
		\d (\bm{X}^{\mathrm{T}}) &= (\d \bm{X})^{\mathrm{T}}
		\label{eq:matrix_derivative_transpose} \\
		\d (\bm{X}^{-1}) &= -\bm{X}^{-1}\d \bm{X}\bm{X}^{-1}
		\label{eq:matrix_derivative_inverse} \\
		\d (\bm{X}\odot \bm{Y}) &= \d \bm{X}\odot \bm{Y} + \bm{X}\odot \d \bm{Y}
		\label{eq:matrix_derivative_hadamard} \\
		\d \left(g(\bm{X})\right) &= g'(\bm{X}) \odot \d \bm{X}
		\label{eq:matrix_derivative_hadamard2}
	\end{align}
\end{subequations}

对于矩阵的迹(trace)运算,有以下常用的性质,在后面的推导中会用到。
\begin{subequations}
	\begin{align}
		a &= \mathrm{tr}(a)
		\label{eq:matrix_trace_scalar} \\
		\mathrm{tr}(\bm{A}^{\mathrm{T}}) &= \mathrm{tr}(\bm{A})
		\label{eq:matrix_trace_transpose} \\
		\mathrm{tr}(\bm{A}\pm\bm{B}) &= \mathrm{tr}(\bm{A}) \pm \mathrm{tr}(\bm{B})
		\label{eq:matrix_trace_add_sub} \\
		\mathrm{tr}(\bm{AB}) &= \mathrm{tr}(\bm{BA})
		\label{eq:matrix_trace_mul} \\
		\mathrm{tr}\left(\bm{A}^{\mathrm{T}}(\bm{B} \odot \bm{C})\right) &= \mathrm{tr}\left((\bm{A} \odot \bm{B})^{\mathrm{T}}\bm{C}\right)
		\label{eq:matrix_trace_hadamard}
	\end{align}
\end{subequations}

在复合情况下,为了避免矩阵对矩阵求导,尽量不要使用链式法则,而是使用微分的性质\eqref{eq:scalar_matrix_derivative_trace},通过迹运算的变形来求解。

\begin{example}\label{example:matrix_derivative1}
% \vspace{0.5\baselineskip}
\colorbox{LemonChiffon}{已知$\dfrac{\partial f}{\partial \bm{Y}}$和$\bm{Y}=\bm{A}\bm{X}\bm{B}$,求$\dfrac{\partial f}{\partial \bm{X}}$}。
根据式\eqref{eq:scalar_matrix_derivative_trace},有
\begin{equation}
	\d f = \mathrm{tr}\left({\frac{\partial f}{\partial \bm{Y}}}^{\mathrm{T}}\d \bm{Y}\right)
	\label{eq:matrix_derivative_example1_1}
\end{equation}
对式\eqref{eq:matrix_derivative_example1_1}中的$\d \bm{Y}$进行展开,有
\begin{equation}
	\d \bm{Y} = \d (\bm{A}\bm{X}\bm{B}) = \d \bm{A}\cdot\bm{X}\bm{B} + \bm{A}\d \bm{X}\cdot\bm{B} + \bm{A}\bm{X}\d \bm{B}
	\label{eq:matrix_derivative_example1_2}
\end{equation}
在$A$, $B$均为常量的情况下,式\eqref{eq:matrix_derivative_example1_2}中的第一项和第三项为零,故式\eqref{eq:matrix_derivative_example1_1}可以化简为
\begin{equation}
	\d f = \mathrm{tr}\left({\frac{\partial f}{\partial \bm{Y}}}^{\mathrm{T}}\bm{A}\d\bm{X}\bm{B}\right)
	\label{eq:matrix_derivative_example1_3}
\end{equation}
根据式\eqref{eq:matrix_trace_mul},式\eqref{eq:matrix_derivative_example1_3}可以进一步改写为
\renewcommand{\eqnhighlightheight}{\vphantom{{\frac{\partial f}{\partial \bm{Y}}}^{\mathrm{T}}}\mathstrut}  % 调整注释框高度便于统一,默认为空
\begin{equation}
	\d f 
	= \mathrm{tr}\left(\eqnmarkbox[blue]{c1}{{\frac{\partial f}{\partial \bm{Y}}}^{\mathrm{T}}\bm{A}\d \bm{X}} \eqnmarkbox[red]{c2}{\bm{B}}\right)
	= \mathrm{tr}\left(\eqnmark[RoyalPurple]{t1}{\bm{B}{\frac{\partial f}{\partial \bm{Y}}}^{\mathrm{T}}\bm{A}} \d \bm{X}\right)
	= \mathrm{tr}\left(\eqnmark[RoyalPurple]{t2}{\left(\bm{A}^{\mathrm{T}}\frac{\partial f}{\partial \bm{Y}}\bm{B}^{\mathrm{T}}\right)^{\mathrm{T}}} \d \bm{X}\right)
	\vspace{1em}
	\label{eq:matrix_derivative_example1_4}
	\tikzset{annotate equations/arrow/.style={color=ForestGreen, >=latex', semithick, dashed}}  % 注释箭头样式,双向绿色虚线箭头,默认为空
	\annotatetwo[yshift=-1em]{below, label below}{c1}{c2}{exchange}
	\tikzset{annotate equations/arrow/.style={->, semithick, dashed}}  % 单向箭头
	\annotatetwo[yshift=-1em]{below, label below}{t1}{t2}{equivalent}
\end{equation}
进而对比式\eqref{eq:scalar_matrix_derivative_trace},可得出
\renewcommand{\eqnhighlightshade}{100}  % 注释框不透明度,默认17
\begin{equation}
	\eqnmarkbox[LemonChiffon]{matrix_derivative_example1}{
	\frac{\partial f}{\partial \bm{X}} = \bm{A}^{\mathrm{T}}\frac{\partial f}{\partial \bm{Y}}\bm{B}^{\mathrm{T}}
	}
	\label{eq:matrix_derivative_example1_5}
\end{equation}
\end{example}

\begin{example}\label{example:matrix_derivative2}
\colorbox{Aquamarine!30}{已知$\dfrac{\partial f}{\partial \bm{Y}}$和$\bm{Y}=g(\bm{X})\odot\bm{X}$,求$\dfrac{\partial f}{\partial \bm{X}}$}。
我们同样有
\begin{equation}
	\d f = \mathrm{tr}\left({\frac{\partial f}{\partial \bm{Y}}}^{\mathrm{T}}\d \bm{Y}\right)
	\label{eq:matrix_derivative_example2_1}
\end{equation}
由式\eqref{eq:matrix_derivative_hadamard2},
\begin{equation}
	\d \bm{Y} = \d \left(g(\bm{X})\right) = g'(\bm{X}) \odot \d \bm{X}
	\label{eq:matrix_derivative_example2_2}
\end{equation}
代入\eqref{eq:matrix_derivative_example2_1},
\begin{equation}
	\d f = \mathrm{tr}\left({\frac{\partial f}{\partial \bm{Y}}}^{\mathrm{T}}\left(\vphantom{A^T}g'(\bm{X}) \odot \d \bm{X}\right)\right)
	\label{eq:matrix_derivative_example2_3}
\end{equation}
根据式\eqref{eq:matrix_trace_hadamard},式\eqref{eq:matrix_derivative_example2_3}可以进一步改写为
\begin{equation}
	\begin{aligned}
		\d f 
		&= \mathrm{tr}\left(\eqnmark[Olive]{}{\frac{\partial f}{\partial \bm{Y}}}^{\mathrm{T}}\left(\vphantom{A^T}\eqnmark[Teal]{}{g'(\bm{X})} \odot \eqnmark[Maroon]{}{\d \bm{X}}\right)\right) \\
		&= \mathrm{tr}\left(\left({\eqnmark[Olive]{}{\frac{\partial f}{\partial \bm{Y}}}\odot \eqnmark[Teal]{}{g'(\bm{X})}}\right)^{\mathrm{T}}\eqnmark[Maroon]{}{\d \bm{X}}\right) \\
		&= \mathrm{tr}\left(\left({\eqnmark[Teal]{}{g'(\bm{X})} \odot \eqnmark[Olive]{}{\frac{\partial f}{\partial \bm{Y}}}}\right)^{\mathrm{T}}\eqnmark[Maroon]{}{\d \bm{X}}\right)
	\end{aligned}
\end{equation}
对比式\eqref{eq:scalar_matrix_derivative_trace},可得出
\renewcommand{\eqnhighlightshade}{30}  % 注释框不透明度,默认17
\begin{equation}
	\eqnmarkbox[Aquamarine]{matrix_derivative_example2}{
	\frac{\partial f}{\partial \bm{X}} = g'(\bm{X}) \odot \frac{\partial f}{\partial \bm{Y}}
	}
	\label{eq:matrix_derivative_example2_4}
\end{equation}
\end{example}

从例~\ref{example:matrix_derivative1} 和例~\ref{example:matrix_derivative2} 可得到下面的结论
\begin{align}
	\bm{Y}=\bm{A}\bm{X}\bm{B} \quad &\Rightarrow \quad \frac{\partial f}{\partial \bm{X}} = \bm{A}^{\mathrm{T}}\frac{\partial f}{\partial \bm{Y}}\bm{B}^{\mathrm{T}}
	\label{eq:matrix_derivative_conclusion1} \\ %%
	\bm{Y}=g(\bm{X})\odot\bm{X} \quad &\Rightarrow \quad \frac{\partial f}{\partial \bm{X}} = g'(\bm{X}) \odot \frac{\partial f}{\partial \bm{Y}} \vphantom{{\frac{\partial f}{\partial \bm{X}}}^{\mathrm{T}}}
	\label{eq:matrix_derivative_conclusion2}  %%
\end{align}
