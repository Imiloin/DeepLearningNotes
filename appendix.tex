\appendix
\chapter{附录}
	
\section{Matrix derivative}
\label{sec:matrix_derivative}
\index{Matrix derivative}

\begin{wenxintishi}
	在本节中,为便于区分,用粗体大写字母表示矩阵,用粗体小写字母表示向量,用普通小写字母表示标量。这些表示与正文不同。
\end{wenxintishi}

矩阵微积分的表示通常有两种符号约定,分别是\textbf{分子布局(Numerator Layout)}和\textbf{分母布局(Denominator Layout)}。下面将分别介绍这两种符号约定。
分子布局的表示方法是将微分符号$\frac{\partial}{\partial x}$放在矩阵的右侧,即$\frac{\partial}{\partial x}\bm{A}$,这种表示方法在\cite{matrix_cookbook}中被称为\textbf{分子布局}。分母布局的表示方法是将微分符号$\frac{\partial}{\partial x}$放在矩阵的左侧,即$\bm{A}\frac{\partial}{\partial x}$,这种表示方法在\cite{matrix_cookbook}中被称为\textbf{分母布局}。

\subsection{标量关于向量的偏导数}
