\chapter{Neural Networks and Deep Learning}

	\section{Introduction to Deep Learning}
	机器学习本质上就是训练模型来完成\textbf{从输入到输出($X\to Y$)的映射}。例如, 给定一张猫的照片, 机器学习模型可以输出``这是一只猫''的结论。

	\vspace{0.5\baselineskip} % 半行距

	深度学习有3大要点:
	\begin{itemize}	
		\item Data
		\item Computation
		\item Algorithms
	\end{itemize}

	Data可以分为Structured Data和Unstructured Data。Structured Data是指有固定格式的数据, 例如表格、数据库等。Unstructured Data是指没有固定格式的数据, 例如图片、音频、视频等。

	Computation是指计算机的计算能力, 例如CPU、GPU、TPU等。

	Algorithms是指算法, 例如Logistic Regression、SVM、Neural Networks等。

	\section{Neural Networks Basics}
	
	\subsection{Binary Classification}

	表 \ref{tab:notations} 是本章用到的符号说明。

	\begin{table}[h]
		\centering
		\begin{threeparttable}
		%%%%%
		\caption{表格标题}
		%
		\begin{tabular}{clcc}
			\hline
									 & \textbf{Notation}                         & \textbf{Description} & \textbf{Meaning}                                                   \\ \hline
			\multirow{5}{*}{Sizes}   & $m$                                       & value                & 样本容量                                                               \\
									 & $n_x$                                     & value                & 样本的特征数量                                                            \\
									 & $n_y$                                     & value                & 输出节点数                                                              \\
									 & $n_h^{[l]}$                               &                      &                                                                    \\
									 & $L$                                       &                      &                                                                    \\ \hline
			\multirow{7}{*}{Objects} & $x^{(i)} \in \mathbb{R}^{n_x}$            & vector               & 第$i$个样本数据                                                          \\
									 & $X \in {\mathbb{R}^{n_x \times m}}$       & matrix               & 输入矩阵                                                               \\
									 & $a^{(i)} / y^{(i)} \in \mathbb{R}^{n_y}$  & vector               & 第$i$个样本的标签                                                         \\
									 & $Y \in {\mathbb{R}^{n_y \times m}}$       & matrix               & 标签矩阵                                                               \\
									 & $\hat{y} \in \mathbb{R}^{n_y}$            & vector               & 单个样本的预测值                                                           \\
									 & $w \in \mathbb{R}^{n_x}$                  & vector               & 权重向量                                                               \\
									 & $W^{[l]}$                                 &                      &                                                                    \\
									 & $b^{[l]}$                                 &                      &                                                                    \\ \hline
			\multirow{2}{*}{Other}   & $J(x,W,b,y)$ or $J(\hat{y},y)$            & function             & 代价函数                                                               \\
									 & $\mathrm{d}\mathrm{var}$                  & differential         & 代价函数对变量$\mathrm{var}$的微分,即${\mathrm{d}J}/{\mathrm{d}\mathrm{var}}$ \\ \hline
		\end{tabular}
		%
		\label{tab:notations} %%%
    	\begin{tablenotes}
      		\item[*] 通常情况下$(i)$表示第$i$个样本,而$[l]$表示第$l$层。
    	\end{tablenotes}
		%%%%%
		\end{threeparttable}
	\end{table}

	\subsection{Logistic Regression}

	对于二元分类模型,相关公式如下:
	\begin{equation}
		\hat{y}^{(i)} = \sigma(w^\mathrm{T} x^{(i)} + b) \label{eq:logistic}
	\end{equation}
	其中sigmoid函数$\sigma(z)$定义为:
	\begin{equation}
		\sigma(z) = \frac{1}{1 + \mathrm{e}^{-z}} \label{eq:sigmoid}
	\end{equation}
	该函数可以把实数域$\mathbb{R}$映射到区间$(0, 1)$,如图 \ref{fig:sigmoid} 所示。而$z = w^\mathrm{T} x + b$为线性函数,$w$为权重(weight),$b$为偏置(bias)。
	\begin{figure}[h]
		\centering
		\includesvg[width=8cm]{sigmoid}
		\caption{Sigmoid 函数,关于点(0, 0.5)对称}
		\label{fig:sigmoid}
	\end{figure}

	\vspace{0.5\baselineskip}
	对某一个样本而言,Loss function(误差函数)定义为:
	\begin{equation}
		L(\hat{y}^{(i)}, y^{(i)}) = -(y^{(i)} \log \hat{y}^{(i)} + (1 - y^{(i)}) \log (1 - \hat{y}^{(i)})) \label{eq:loss}
	\end{equation}
	该函数与方差类似,当样本的误差越大,Loss function的值越大,如图 \ref{fig:loss} 所示。
	\begin{figure}[h]
		\centering
		\includesvg[width=8cm]{LossFunction}
		\caption{Loss Function 误差函数}
		\label{fig:loss}
	\end{figure}

	对整个模型而言,Cost function(代价函数)定义为:
	\begin{equation}
		J(w, b) = \frac{1}{m} \sum_{i=1}^{m} L(\hat{y}^{(i)}, y^{(i)}) = -\frac{1}{m} \sum_{i=1}^{m} (y^{(i)} \log \hat{y}^{(i)} + (1 - y^{(i)}) \log (1 - \hat{y}^{(i)})) \label{eq:cost}
	\end{equation}
	相当于误差函数的平均值。

	\subsection{Gradient Descent}

	训练过程使用梯度下降(Gradient Descent)算法,即:
	\begin{equation}
		\mathrm{var} = \mathrm{var} - \alpha \frac{\mathrm{d}J}{\mathrm{d}\mathrm{var}} \label{eq:gradient}
	\end{equation}

	其中$\mathrm{var}$为一个需要调整的参数,$\alpha$为学习率(Learning Rate)。对于Logistic Regression,梯度下降算法的表达式为:
	\begin{equation}
		\begin{aligned}
		w &:= w - \alpha \frac{\mathrm{d}J}{\mathrm{d}w} \\
		b &:= b - \alpha \frac{\mathrm{d}J}{\mathrm{d}b}
		\end{aligned} 
		\label{eq:gradient_logistic}
	\end{equation}
	其中$:=$表示赋值,通过不断更新$w$, $b$的值使$J$尽可能小。下面介绍具体实现

	\vspace{0.5\baselineskip}
	对于一个样本,有
	\begin{equation}
		\begin{aligned}
		z &= w^\mathrm{T} x^{(i)} + b \\
		\hat{y}^{(i)} &= \sigma(z) \\
		L(\hat{y}^{(i)}, y^{(i)}) &= -(y^{(i)} \log \hat{y}^{(i)} + (1 - y^{(i)}) \log (1 - \hat{y}^{(i)}))
		\end{aligned} 
		\label{eq:gradient_logistic_sample}
	\end{equation}
	进行求导($x$, $y$均为常量),有
	\begin{equation}
		\begin{aligned}
		&\frac{\mathrm{d}L}{\mathrm{d}\hat{y}^{(i)}} = -\frac{y^{(i)}}{\hat{y}^{(i)}} + \frac{1 - y^{(i)}}{1 - \hat{y}^{(i)}} \\
		&\frac{\mathrm{d}\hat{y}^{(i)}}{\mathrm{d}z} = \frac{\mathrm{e}^{-z}}{(1 + \mathrm{e}^{-z})^2} = \frac{1}{1 + \mathrm{e}^{-z}}(1 - \frac{1}{1 + \mathrm{e}^{-z}}) = \hat{y}^{(i)}(1 - \hat{y}^{(i)})\\
		\end{aligned}
	\end{equation}
	用链式法则,有
	\begin{equation}
		\frac{\mathrm{d}L}{\mathrm{d}z} = \frac{\mathrm{d}L}{\mathrm{d}\hat{y}^{(i)}} \frac{\mathrm{d}\hat{y}^{(i)}}{\mathrm{d}z} = (-\frac{y^{(i)}}{\hat{y}^{(i)}} + \frac{1 - y^{(i)}}{1 - \hat{y}^{(i)}})\hat{y}^{(i)}(1 - \hat{y}^{(i)}) = \hat{y}^{(i)} - y^{(i)}
	\end{equation}
	进而有
	\begin{equation}
		\begin{aligned}
		&\frac{\mathrm{d}L}{\mathrm{d}w_j} = \frac{\mathrm{d}L}{\mathrm{d}z} \frac{\mathrm{d}z}{\mathrm{d}w_j} = x_j^{(i)} (\hat{y}^{(i)} - y^{(i)}) \quad (1 \leqslant j \leqslant n_x) \\
		&\frac{\mathrm{d}L}{\mathrm{d}b} = \frac{\mathrm{d}L}{\mathrm{d}z} \frac{\mathrm{d}z}{\mathrm{d}b} = \hat{y}^{(i)} - y^{(i)}
		\end{aligned}
	\end{equation}

	\vspace{0.5\baselineskip}
	对于整个训练集,在一个循环中我们对每一个样本都进行一次梯度下降,并对所有样本的梯度进行平均,有
	\begin{equation}
		\begin{aligned}
			\frac{\mathrm{d}J}{\mathrm{d}w_j} &= \frac{1}{m} \sum_{i=1}^{m} \frac{\mathrm{d}L}{\mathrm{d}w_j} = \frac{1}{m} \sum_{i=1}^{m} x_j^{(i)} (\hat{y}^{(i)} - y^{(i)}) \quad (1 \leqslant j \leqslant n_x)\\
			\frac{\mathrm{d}J}{\mathrm{d}b} &= \frac{1}{m} \sum_{i=1}^{m} \frac{\mathrm{d}L}{\mathrm{d}b} = \frac{1}{m} \sum_{i=1}^{m} (\hat{y}^{(i)} - y^{(i)})
		\end{aligned}
	\end{equation}
	得到梯度的平均值后,再进行参数更新,即
	\begin{equation}
		\begin{aligned}
		w_j &:= w_j - \alpha \frac{\mathrm{d}J}{\mathrm{d}w} \quad (1 \leqslant j \leqslant n_x) \\
		b &:= b - \alpha \frac{\mathrm{d}J}{\mathrm{d}b}
		\end{aligned} 
	\end{equation}
	这样就完成了一轮梯度下降的循环。重复多次梯度下降,直到$J$收敛或达到最大迭代次数,就得到了最终的$w$, $b$。

	\subsection{Vectorization}

	一个重要的原则是,在训练过程中要避免使用循环,而是使用向量化的方法。

