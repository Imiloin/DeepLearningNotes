\chapter{Neural Networks and Deep Learning}
	各章开头可有说明文字.\footnote{脚注测试} 视情况加入阅读提示.
	\begin{wenxintishi}
		本章 \S\ref{sec:words} 的文字撷取自《明儒学案》, [清] 黄宗羲, 仅作测试之用. 来源于网络, 恐怕多有错漏, 请见谅. \index{huangzongxi@黄宗羲}
	\end{wenxintishi}

	\section{Introduction to Deep Learning}
	机器学习本质上就是训练模型来完成\textbf{从输入到输出($X\to Y$)的映射}。例如, 给定一张猫的照片, 机器学习模型可以输出``这是一只猫''的结论。

	\vspace{0.5\baselineskip} % 半行距

	深度学习有3大要点:
	\begin{itemize}	
		\item Data
		\item Computation
		\item Algorithms
	\end{itemize}

	Data可以分为Structured Data和Unstructured Data。Structured Data是指有固定格式的数据, 例如表格、数据库等。Unstructured Data是指没有固定格式的数据, 例如图片、音频、视频等。

	\section{Neural Networks Basics}
	
	

	\section{文字测试}\label{sec:words}

	测试 \textsf{description} 环境如下:
	\begin{description}
		\item[明儒学案] 中国第一部完整的学术史著作
		\item[黄宗羲 (1610---1695)] 明末清初思想家、文学家。字太冲,号梨洲,又号南雷。
	\end{description}
	
	\subsection{子节 (subsection)}
	以下测试定义, 定理, 证明等等, 以及文字加粗.
	
	\begin{definition}
		盈天地皆心也,变化不测,不能不万殊。\emph{心无本体,工夫所至,即其本体},故穷理者,穷此心之万殊,非穷万物之万殊也。是以古之君子,宁凿五丁之间道,不假邯郸之野马,故其途亦不得不殊!奈何今之君子,必欲出於一途,使美厥灵根者,化为焦芽绝港。夫先儒之语录,人人不同,只是印我之心体,变动不居,若执定成局,终是受用不得。此无他,修德而后可讲学。今讲学而不修德,又何怪其举一而废百乎?
	\end{definition}

	\subsection*{无编号子节 (subsection*)}
	无编号者不列入目录.
	\begin{proposition}[胡居仁]
		胡居仁字叔心,饶之余干人也。学者称为敬斋先生。弱冠时奋志圣贤之学,往游康斋吴先生之门,遂绝意科举,筑室於梅溪山中,事亲讲学之外,不干人事。久之,欲广闻见,适闽历浙、入金陵,从彭蠡而返。所至访求问学之士,归而与乡人娄一斋、罗一峰、张东白为会於弋阳之龟峰、余干之应天寺。提学李龄、锺城相继请主白鹿书院。诸生又请讲学贵溪桐源书院。淮王闻之,请讲《易》於其府。王欲梓其诗文,先生辞曰:“尚需稍进。”先生严毅清苦,左绳右矩,每日必立课程,详书得失以自考,虽器物之微,区别精审,没齿不乱。
	\end{proposition}

	\subsubsection{次子节 (subsubsection)}
	次子节默认不再编号. 如需编号, 请手动设置 \LaTeX 中标准的 \textsf{secnumdepth} 参数.

	\begin{lemma}[陈献章]\label{prop:chen}
		有明之学,至白沙始入精微。其吃紧工夫,全在涵养。喜怒未发而非空,万感交集而不动,至阳明而后大。两先生之学,最为相近,不知阳明后来从不说起,其故何也?薛中离,阳明之高第弟子也,於正德十四年上疏请白沙从祀孔庙,是必有以知师门之学同矣。罗一峰曰:“白沙观天人之微,究圣贤之蕴,充道以富,崇德以贵,天下之物,可爱可求,漠然无动於其中。”信斯言也,故出其门者,多清苦自立,不以富贵为意,其高风之所激,远矣。
	\end{lemma}
	\begin{proof}
		陈献章字公甫,新会之白沙里人。身长八尺,目光如星,右脸有七黑子,如北斗状。自幼警悟绝人,读书一览辄记。尝读《孟子》所谓天民者,慨然曰:“为人必当如此!”梦拊石琴,其音泠泠然,一人谓之曰:“八音中惟石难谐,子能谐此,异日其得道乎?”因别号石斋。正统十二年举广东乡试,明年会试中乙榜,入国子监读书。已至崇仁,受学於康斋先生,归即绝意科举,筑春阳台,静坐其中,不出阈外者数年。寻遭家难。成化二年,复游太学,祭酒邢让试和杨龟山《此日不再得》诗,见先生之作,惊曰:“即龟山不如也。”扬言於朝,以为真儒复出,由是名动京师。罗一峰、章枫山、庄定山、贺医闾皆恨相见之晚,医闾且禀学焉。归而门人益进。十八年,布政使彭韶、都御史朱英交荐,言“国以仁贤为宝,臣自度才德不及献章万万,臣冒高位,而令献章老丘壑,恐坐失社稷之宝”。召至京,阁大臣或尼之,令就试吏部。辞疾不赴,疏乞终养,授翰林院检讨而归。有言其出处与康斋异者,先生曰:“先师为石亨所荐,所以不受职;某以听选监生,始终愿仕,故不敢伪辞以钓虚誉,或受或不受,各有攸宜。”自后屡荐不起。弘治十三年二月十日卒,年七十有三。先生疾革,知县左某以医来,门人进曰:“疾不可为也。”先生曰:“须尽朋友之情。”饮一匙而遣之。
	\end{proof}

	\paragraph{段落 (paragraph)} 段落一般也不编号.
	\begin{corollary}[吕柟]
		字仲木,号泾野,陕之高陵人。正德戊辰举进士第一,授翰林修撰。逆瑾以乡人致贺,却之,瑾不悦。已请上还宫中,御经筵,亲政事,益不为瑾所容,遂引去。瑾败,起原官。上疏劝学,危言以动之。乾清宫灾,应诏言六事:一、逐日临朝,二、还处宫寝,三、躬亲大祀,四、日朝两宫,五、遣去义子、番僧、边军,六、撤回镇守中官。皆武宗之荒政。不听,复引去。世庙即位,起原官。甲申以修省自劾,语涉大礼,下诏狱。降解州判官,不以迁客自解,摄守事,兴利除害若嗜欲。
	\end{corollary}
	\begin{proof}
		未第时,即与崔仲凫讲於宝邛寺。正德末,家居筑东郭别墅,以会四方学者。别墅不能容,又筑东林书屋。镇守廖奄张甚,其使者过高陵,必诫之曰:“吕公在,汝不得作过也。”在解州建解梁书院,选民间俊秀,歌诗习礼。九载南都,与湛甘泉邹东廓共主讲席,东南学者,尽出其门。尝道上党,隐士仇栏遮道问学。有梓人张提闻先生讲,自悟其非,曾妄取人物,追还主者。先生因为诗云:“岂有征夫能过化,雄山村里似尧时。”朝鲜国闻先生名,奏谓其文为式国中。先生之学,以格物为穷理。及先知而后行,皆是儒生所习闻。而先生所谓穷理,不是泛常不切於身,只在语默作止处验之;所谓知者,即从闻见之知,以通德性之知,但事事不放过耳。大概工夫,下手明白,无从躲闪也。
	\end{proof}

	\begin{remark}
		诸生有言及气运如何,外边人事如何者。曰:“\emph{此都是怨天尤人的心术}。但自家修为,成得个片段,若见用,则百姓受些福;假使不用,与乡党朋友论些学术,化得几人,都是事业,正所谓畅於四肢,发於事业也,何必有官做,然后有事业。” 
	\end{remark}

	\section{字体和大小}
	自带的设定档中定义了中文排版常用的几种字体命令, 可以手动切换. 如表 \ref{table:ziti}.
	\begin{table}[h!]
		\begin{tabular}{ll}
			\texttt{\textbackslash heiti} & {\heiti 黑体} \\
			\texttt{\textbackslash songti} & {\songti 宋体} \\
			\texttt{\textbackslash kaishu} & {\kaishu 楷体} \\
			\texttt{\textbackslash fangsong} & {\fangsong 仿宋}
		\end{tabular}
		\caption{几种字体命令} \label{table:ziti}
	\end{table}

	\emph{注意}: \LaTeX 的精神是尽量让作者专注于内容, 外观则留给模板. 频繁地手动切换字体不是个好主意.
	
	字体大小由标准命令控制, 如表 \ref{table:ziti-size} 所示.

	\begin{table}[h!]
		\begin{tabular}{ll}
			\texttt{\textbackslash tiny} & {\tiny 极高明而道中庸} \\
			\texttt{\textbackslash scriptsize} & {\scriptsize 极高明而道中庸} \\
			\texttt{\textbackslash footnotesize} & {\footnotesize 极高明而道中庸} \\
			\texttt{\textbackslash small} & {\small 极高明而道中庸} \\
			\texttt{\textbackslash normalsize} & {\normalsize 极高明而道中庸} \\
			\texttt{\textbackslash large} & {\large 极高明而道中庸} \\
			\texttt{\textbackslash Large} & {\Large 极高明而道中庸} \\
			\texttt{\textbackslash LARGE} & {\LARGE 极高明而道中庸} \\
			\texttt{\textbackslash huge} & {\huge 极高明而道中庸} \\
			\texttt{\textbackslash Huge} & {\Huge 极高明而道中庸}
		\end{tabular}
		\caption{字体大小效果} \label{table:ziti-size}
	\end{table}

	\section{图片}
	
	本模板采用\emph{知识共享}署名 4.0 国际许可协议进行许可. 点击\href{http://creativecommons.org/licenses/by/4.0/}{链接}查看该许可协议.
	\begin{figure}[h!]\begin{center}
		\includegraphics{ccby.png}
		\caption{许可协议图片}
	\end{center}\end{figure}

	\section{混排公式 \texorpdfstring{$e^{i\theta} = \cos\theta + i\sin\theta$}{exp itheta = cos theta + isin theta}}
	西文按寻常方式进行排版.
	\begin{quote}
		La connaissance de la misère humaine est difficile au riche, au puissant, parce qu'il est presque invinciblement porté à croire qu'il est quelque chose. Elle est également difficile au misérable parce qu'il est presque invinciblement porté à croire que le riche, le puissant est quelques chose.

		\begin{flushright}\textit{La Pesanteur et la grâce}, Simone Weil\end{flushright}
	\end{quote}

	每章最后可以集中安排习题.
	\begin{Exercises}
		\item 试证......
		\begin{hint}
			请自己做.
		\end{hint}
		\item 试说明在一般的环 $R$ 中
		\begin{enumerate}
			\item 对所有 $x, y \in R$ 皆有 $x + y = y + x$;
			\item 但一般而言
			\[ xy \neq yx. \]
		\end{enumerate}
	\end{Exercises}